\begin{enumerate}
    \item $\absTxNum$: transaction number column, starts at 0 and is incremented by one when starting a new transaction.
    \item $\absTxNumInfty$: batch constant column, indicating the number of transaction in the batch;
    \item $\absLogNum$: number of the current log entry, only used in $\phase{5}$;
    \item $\absLogNumMax$: number of log entries in a given transaction;
    \item Output columns:
        \begin{enumerate}
            \item $\limb$: output of the module, contains $\llarge$ byte integers left aligned on the $\llarge$-th byte\footnote{ and thus with $\llarge - \limbsize$ zeros occupying the least significant bytes}
            \item $\limbsize$: number of meaningful bytes of the $\limb$
            \item \col{LIMB\_CONSTRUCTED} abbreviated $\lc$: binary column.
            \item $\index$: index of the current limb within the current string; starts at 0 when starting a new transaction and increments by 1 in the row \emph{following} $\lc = 1$.
            \item $\indexlocal$: local index, used to reconstruct data $O_{\mathbf{d}}$ and the 256 bytes hash $R_{\mathrm{b}}$.
        \end{enumerate} 
    \item Columns used for heartbeat
        \begin{enumerate}
            \item \phase{1} to \phase{5}: 5 binary columns that indicates the phase.
            \item $\phasend$: binary column. Is one at the last row of a phase, 0 elsewhere.
            \item $\col{COUNTER}$, abbreviated $\ct$: internal counter, byte column. Counts from 0 to $\nbstep - 1$ and resets.
            \item $\nbstep$: byte column, with value between 1 and $\llarge$.
            \item $\Done$: binary column to be one when at the end of a \ct-loop, else 0.
        \end{enumerate}
    \item Columns used in several phase for computation
        \begin{enumerate}
            \item $\logsize$: Byte size of the log, must be 0 at the end of the transaction.
            \item $\Input k$, $k=1,2, 3, 4$: 0-16 bytes data.
            \item $\byteCol{k}$, $k=1,2, 3, 4$: byte columns used for byte decompositions, and its accumulators $\acc{k}$.
            \item $\accsize$: byte column used to count the bytesize of $\acc{1}$.
            \item $\col{BIT}$: binary column.
            \item $\col{BIT\_ACC}$, bit accumulator column.
            \item $\Power$, a power of 256, to calculate the offset (left shifting) for the $\limb$.
            \item \isprefix: binary column, to declare when computing a \rlp{} prefix ($\phase{4}$ and $\phase{5}$) or concatenating the type of the transaction ($\phase{1}$).
            \item \ispadding: binary column; equals $1$ \emph{iff} the current \ct{}-loop produces no limb;
        \end{enumerate}
        \saNote{} \ispadding{} can only be on during phase 1 or phase 5: for phase 1 it turns on for transactions of type 0 (which require not prepending of a transaction type byte); during phase 5 it turns on whenever the data consists of a single byte in the range $\texttt{0x\,00} < \texttt{b} < \texttt{0x\,80}$.
    \item Columns used in phase 4 and 5 only
        \begin{enumerate}
            \item $\phasesize$: size (in bytes) of the bytestring of this phase; used to compute the \rlp{} prefix of the byte string of this phase;  decrements as the module accounts for these bytes;
            % \item $\isEightByteInput$: binary column used to distinguish between 8 byte inputs for 
        \end{enumerate}
    \item Columns used in phase 5 only
        \begin{enumerate}
            \item $\Depth1$: binary column to define loop intrication.
            \item $\isOt$: binary column to define when dealing with $O_{\mathbf{t}}$.
            \item $\isOd$: binary column to define when dealing with $O_{\mathbf{d}}$.
            \item $\logentrysize$: byte size of a \rlp{}-ization of log entry; decrements as the module accounts for these bytes;
            \item $\localsize$: byte size of either the data being logged in a log entry or of the size of concatenation of the \rlp{}-izations of the log topics of a log item; decrements as the module accounts for these bytes;
        \end{enumerate}
    \item \Phase{}: column used as an identifier for the lookups;

\end{enumerate}
\saNote{} \textbf{All sizes} referred to by the columns of the present module are measured in \textbf{bytes}. 
