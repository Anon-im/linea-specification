Note that, for $k=1,=2$, the condition $\instSigned_{i} \cdot \col{sgn\_k} = 1$ is equivalent to
$\instSigned_{i} = 1$ and 
$\col{sgn\_k} = 1$
i.e. the current instruction is a signed instruction (\inst{SMOD} or \inst{SDIV}) 
and its $k$-th argument represents a negative integer. With this and our discussion about absolute values in 
section~\ref{alu: mod: absolute values} we impose
\begin{enumerate}
	\item $\setAbsoluteValue
		\left( \begin{array}{l}
			\absOneHi, \absOneLo;              \\
			\argOneHi_{i}, \argOneLo_{i};      \\
			\instSigned_{i} \cdot \col{sgn\_1} \\
		\end{array} \right)$
	\item $\setAbsoluteValue
		\left( \begin{array}{l}
			\absTwoHi, \absTwoLo;              \\
			\argTwoHi_{i}, \argTwoLo_{i};      \\
			\instSigned_{i} \cdot \col{sgn\_2} \\
		\end{array} \right)$
\end{enumerate}
\saNote{}
Both pairs, $\absOneHi$, $\absOneLo$ as well as $\absTwoHi$, $\absTwoLo$,
will be explicitly defined in section~(\ref{alu: mod: target}),
see note~(\ref{alu: mod: target: implicit definition of A/B_HI/LO}).
The discussion~(\ref{alu: primitives: absolute values: hypotheses and conclusion})
entails the satisfaction of \textbf{the above constraints uniquely pin down both pairs}.
