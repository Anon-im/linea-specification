For the sake of clarity and connecting the constraints of this section with the strategy for proving a euclidean division outlined in section~(\ref{subsec (alu/mod): verif}) we introduce some \emph{aliases} for certain \emph{values}. We emphasize that these don't represent new columns. They are only meant to capture the value of the accumulators at the final row of the byte accumulation process i.e. rows with $\ct_{i} = \mmediumMO$. They thus represent the target values these accumulators strive for.
\begin{multicols}{3}
\begin{enumerate}
	\item $\col{ARG1\_3} \define \acc{1\_3}_{i}$
	\item $\col{ARG1\_2} \define \acc{1\_2}_{i}$
	\item $\col{ARG2\_3} \define \acc{2\_3}_{i}$
	\item $\col{ARG2\_2} \define \acc{2\_2}_{i}$
	\item $\col{B\_\red{k}} \define \acc{B\_\red{k}}_{i}$
	\item $\col{Q\_\red{k}} \define \acc{Q\_\red{k}}_{i}$
	\item $\col{R\_\red{k}} \define \acc{R\_\red{k}}_{i}$
	\item $\col{$\Delta$\_\red{k}} \define \acc{$\Delta$\_\red{k}}_{i}$
	\item $\col{R\_HI} \define \theta \cdot \col{R\_3} + \col{R\_2}$
	\item $\col{R\_LO} \define \theta \cdot \col{R\_1} + \col{R\_0}$
	\item $\col{Q\_HI} \define \theta \cdot \col{Q\_3} + \col{Q\_2}$
	\item $\col{Q\_LO} \define \theta \cdot \col{Q\_1} + \col{Q\_0}$
	\item $\col{lt\_\red{k}} \define \cmp{1}_{i - 7 + \red{k}}$
	\item $\col{eq\_\red{k}} \define \cmp{2}_{i - 7 + \red{k}}$
	\item $\col{sgn\_1} \define \msb{1}_{i - 7}$
	\item $\col{sgn\_2} \define \msb{2}_{i - 7}$
	\item $\col{H\_\red{n}} \define \acc{H\_\red{n}}_{i}$
	\item $\alpha \define \cmp{2}_{i - 3}$
	\item $\beta\col{\_0} \define \cmp{2}_{i - 2}$
	\item $\beta\col{\_1} \define \cmp{2}_{i - 1}$
	\item $\beta \define 2 \cdot \beta\col{\_1} + \beta\col{\_0}$
\end{enumerate}
\end{multicols}
\noindent where $\col{\red{k}}\in\{0,1,2,3\}$ and $\col{\red{n}}\in\{0,1,2\}$.

We define four more aliases
$\absOneHi$, $\absOneLo$ and
$\absTwoHi$, $\absTwoLo$,
see section~(\ref{alu: mod: target}).
Their definition is more involved thus we don't reproduce it here.
