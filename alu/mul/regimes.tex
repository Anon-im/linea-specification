The heartbeat of the multiplication module is more complex than that of the other arithmetic modules. The explanation lies in the fact that the present module has different ``regimes'' (with the ``(nontrivial) \inst{EXP} regime'' being split into 3 ``sub-regimes.'') There are the two following ``trivial'' regimes
\begin{enumerate}
	\item[0.] The \textbf{padding regime} is characterized by $\mulStamp \equiv 0$; nothing of import happens.
	\item[1.] The \textbf{trivial regime} is characterized by $\oli      \equiv 1$.
		In this regime the result is obvious on sight and a single line suffices to write it.
\end{enumerate}
The following are more involved
\begin{enumerate}
	\item[2.] The \textbf{nontrivial \inst{MUL} regime} is characterized by $\oli \equiv 0$ \et $\INST \equiv \inst{MUL}$. This regime occupies a single counter-cycle
	\item[3.] The \textbf{nontrivial \inst{EXP} regime} applies in the remaining case i.e. $\oli \equiv 0$ \et $\INST \equiv \inst{EXP}$. It is split into three different ``sub-regimes''.
		\begin{enumerate}
			\item 
				The \textbf{zero result regime} is characterized by $\res \equiv 0$ i.e. $\resVanishes \equiv 1$.
				Given that we are dealing with a \emph{nontrivial} exponentiation (since $\oli \neq 1$) this happens \emph{iff}
				\[ \nu_2 \cdot \argTwo \geq 256 \]
				where we define $\nu_2 := \min\Big\{2\text{-adicity of the base }\argOne, \oneTwoEight \Big\}$. This case can be justified in a single counter-cycle (i.e. in $\mmedium$ rows.)
			\item
				The \textbf{nonzero result regime} is characterized by $\res \not\equiv 0$ i.e. $\resVanishes \equiv 0$.
				There are two subcases:
				(\emph{a}) that of a \emph{(relatively) small} exponent $\argTwo$ i.e. $\argTwoHi = 0$;
				(\emph{b}) that of a \emph{large} exponent $\argTwo$ i.e. $\resVanishes_{i} = 0$ \et $\argTwoHi \neq 0$.
				In the first case the result is computed as a sequence of $1 \leq k \leq \oneTwoEight$ ``square and multiply's''.
				In the second case the result is computed as a sequence of $k + \oneTwoEight$ ``square and multiply's'', for some $1 \leq k \leq \oneTwoEight$, 
		\end{enumerate}
\end{enumerate}

Note that each ``square and multiply'' occupies $\mmedium$ or $2\cdot\mmedium$ rows (depending on whether one only ``squares'' or also ``multiplies.'')
