\begin{center}
\framebox{%
All constraints in this section are written under the assumption that
$\begin{cases}
\INST_{i} = \inst{EXP} \\
% \oli_{i} = 0 \\ % useless given that we also impose \ct_{i} = \mmediumMO
\oli_{i} = 0 \\
\resVanishes_{i} = 1 \\
\end{cases}$}
\end{center}

\paragraph{Target constraints.} We impose the following regardless of anything else:
\begin{enumerate}
	\item \If $\ct_{i} = \mmediumMO$ \Then 
	\[
	\begin{cases}
		\argOneHi_{i} = \theta \cdot \col{A\_3} + \col{A\_2} \\
		\argOneLo_{i} = \theta \cdot \col{A\_1} + \col{A\_0} \vspace{2mm} \\
		\argTwoHi_{i} = \theta \cdot \col{B\_3} + \col{B\_2} \\
		\argTwoLo_{i} = \theta \cdot \col{B\_1} + \col{B\_0} \\
	\end{cases}
	\]
	i.e. we enforce
	$\argOne \equiv
	[\col{A\_3} \,|\, \col{A\_2} \,|\, \col{A\_1} \,|\, \col{A\_0} ]$ and
	$\argTwo \equiv
	[\col{B\_3} \,|\, \col{B\_2} \,|\, \col{B\_1} \,|\, \col{B\_0} ]$.
\end{enumerate}

\paragraph{Remark.} Note that if $\argOneLo = 0$ the result of the \inst{EXP} instruction is necessarily zero.
Indeed $\argOneLo = 0$ is equivalent to $2^\oneTwoEight \mid \argOne$ i.e. $\nu_{2} \geq \oneTwoEight$. Furthermore $\oli_{i} = 0$ forces $\argTwo \geq 2$. The desired bound ($\nu_{2} \cdot \argTwo \geq 256$) is thus necessarily achieved. Thus
\begin{center}
\framebox{%
All constraints below make the further assumption that $\argOneLo_{i} \neq 0$.
}
\end{center}

\paragraph{Special constraints for $\col{BYTE\_C\_0}$.} The $\col{BYTE\_C\_0}$ column serves an entirely different purpose from elsewhere in the module:
\begin{enumerate}[resume]
	\item \If $\ct_{i} \neq \mmediumMO$ \Then $\byteCol{C\_0}_{i} = \byteCol{C\_0}_{i + 1}$
	\item \If $\ct_{i} = \mmediumMO$ \Then
	\[
		\begin{cases}
			\If \col{A\_0} =    0 ~ \Then \col{BYTE\_C\_0} = 0 \\
			\If \col{A\_0} \neq 0 ~ \Then \col{BYTE\_C\_0} = 1 \\
		\end{cases}
	\]
\end{enumerate}
\noindent We have thus constrained $\byteCol{C\_0}$:
\begin{itemize}
	\item to be constant along the present counter-cycle;
	\item to be binary along the present counter-cycle;
	\item to retain the information whether $2^{8\cdot\mmedium} \mid \argOne$ or not.
\end{itemize}
Indeed, the previous set of target constraints enforces $\argOne \equiv
[\col{A\_3} \,|\, \col{A\_2} \,|\, \col{A\_1} \,|\, \col{A\_0} ]$.
Therefore
$2^{8\cdot\mmedium} \mid \argOne
\iff \col{A\_0} = 0
\iff \col{BYTE\_C\_0} = 0$.
More precisely (and given that $\argOneLo \neq 0$) we see that
if $\col{A\_0} = 0$ then $\nu_{2} = 8\cdot\mmedium + \text{(the $2$-adicity of $\col{A\_1}$)}$
while 
if $\col{A\_0} \neq 0$ then $\nu_{2} = \text{(the $2$-adicity of $\col{A\_0}$.)}$ This should explain the next set of constraints:
% Thus $\col{BYTE\_C\_0} = 0$ \emph{iff} $\col{A\_0} = 0$. column to be a constant bit along the present counter-cycle and retain the information whether $\col{A}_{0}$ is zero or not, where we write 

\paragraph{Preparation of a lower bound.}
The constraints in this paragraph allow us to extract parameters which are later assembled into a lower bound for the $2$-adicity of the base $\argOne$.
\begin{enumerate}[resume]
	\item
	\If $\byteCol{C\_0}_{i} = 1$\footnote{Given the above this is equivalent to $\byteCol{C\_0}_{i} \neq 0$, which one may choose instead for the implementation.} \Then
	\[
			\prepareLowerBoundOnTwoAdicity
			\left(
			\begin{array}{c}
			\byteCol{A\_0},
			\byteCol{C\_1},
			\bits; \\
			\byteCol{C\_3},
			\byteCol{H\_3}; \\
			\byteCol{C\_2},
			\byteCol{H\_2};
			\ct{};
			\end{array}
			\right) \\
	\]
	\item \If $\byteCol{C\_0}_{i} = 0$\footnote{Given the above this is equivalent to $\byteCol{C\_0}_{i} \neq 1$, which one may choose instead for the implementation.} \Then
	\[
			\prepareLowerBoundOnTwoAdicity
			\left(
			\begin{array}{c}
			\byteCol{A\_1},
			\byteCol{C\_1},
			\bits; \\
			\byteCol{C\_3},
			\byteCol{H\_3}; \\
			\byteCol{C\_2},
			\byteCol{H\_2};
			\ct{};
			\end{array}
			\right)
	\]
\end{enumerate}

\paragraph{Remarks.}
Note that some columns
%$\byteCol{C\_1}$,
%$\byteCol{C\_2}$,
%$\byteCol{H\_2}$,
%$\byteCol{C\_3}$,
%$\byteCol{H\_3}$
were completely repurposed from their standard use case.
In all other cases
$\byteCol{C\_0}$, 
$\byteCol{C\_1}$, 
$\byteCol{C\_2}$, 
$\byteCol{H\_2}$, 
$\byteCol{C\_3}$ and
$\byteCol{H\_3}$
are used to establish some product $\mod 2^{256}$. Here they are repurposed as binary columns, stores for particular pivot bytes etc \dots{} More precisely:
\begin{itemize}
	\item $\byteCol{C\_0}$ is repurposed as a constant bit that distinguishes between the two cases $2^{\mmedium \cdot 8} \mid \argOneLo_{i}$ (i.e. $\byteCol{C\_0} = 0$) and $2^{8\cdot\mmedium} \nmid \argOneLo_{i}$ (i.e. $\byteCol{C\_0} = 1$);
	\item $\byteCol{C\_1}$ is repurposed as a constant column that contains one of the bytes of either
	$\col{A\_1}$ (if $\byteCol{C\_0} = 0$) or
	$\col{A\_0}$ (if $\byteCol{C\_0} = 1$)
	beyond which all remaining bytes of that $\mmedium$-byte integer are zero;
	\item $\bits$ contains the bit decomposition of the constant value in $\byteCol{C\_1}$;
	\item $\byteCol{C\_3}$ is repurposed as a nondecreasing binary column indicating \emph{a} cut-off point beyond which the bytes from either
	$\col{A\_1}$ (if $\byteCol{C\_0} = 0$) or
	$\col{A\_0}$ (if $\byteCol{C\_0} = 1$) are zero;
	\item $\byteCol{H\_3}$ is repurposed as its running total;
	\item $\byteCol{C\_2}$ is repurposed as a nondecreasing binary column indicating \emph{a} cut-off point beyond which the bits of $\bits$ are zero;
	\item $\byteCol{H\_2}$ is repurposed as its running total;
\end{itemize}
Note that using values found in the final row of the present counter-cycle (i.e. the row with $\ct_{i} = \mmediumMO$) we can construct a lower bound \col{lb} for the $2$-adicity of $\argOne$. Indeed
\[
	\col{lb} := 
	\begin{cases}
		\If \byteCol{C\_0}_{i} = 1: & 8 \displaystyle \cdot \byteCol{H\_3}_{i} + \byteCol{H\_2}_{i} \vspace{2mm}\\
		\If \byteCol{C\_0}_{i} = 0: & 8 \displaystyle \cdot \byteCol{H\_3}_{i} + \byteCol{H\_2}_{i} + \mmedium \cdot 8
	\end{cases}
\]
is such a lower bound.
\iffalse
\begin{itemize}
	\item \If $\byteCol{C\_0}_{i} = 1$ \Then $8 \cdot \byteCol{H\_3}_{i} + \byteCol{H\_2}_{i}$ is a lower bound on the $2$-adicity of $\argOne$;
	\item \If $\byteCol{C\_0}_{i} = 0$ \Then $8 \cdot \byteCol{H\_3}_{i} + \byteCol{H\_2}_{i} + \mmedium \cdot 8$ is a lower bound on the $2$-adicity of $\argOne$.
\end{itemize}
\fi

\paragraph{Proving the vanishing of \inst{EXP}.} Now that we have a lower bound (which can be an exact bound if we so choose) we are in position to verify that $\nu_{2} \cdot \argTwo \geq 256$ (or more precisely $k \cdot \argTwo \geq 256$ for our lower bound $k$.)
Note the following: assuming (as we shortly will) that $\ct_{i} = \mmediumMO$) one has
\[
	\argTwo \geq 256 \iff \theta \cdot \Big(\col{B\_3} + \col{B\_2} + \col{B\_1}\Big)
	+ \col{B\_0}
	\neq
	\byteCol{B\_0}_{i}
\]
Indeed
given that
$\argTwo \equiv [\col{B\_3} \,|\, \col{B\_2} \,|\, \col{B\_1} \,|\, \col{B\_0} ]$
it follows that
$\argTwo \geq 256$ \emph{iff} one of the $\mmedium$-byte integers $\col{B\_3}$, $\col{B\_2}$, $\col{B\_1}$ is $\neq 0$ or they are all $=0$ so that $\argTwo = \col{B\_0}$ but (the $\mmedium$-byte integer) $\col{B\_0}$ isn't a byte i.e. (since $\byteCol{B\_0}$ contains its byte decomposition by virtue of section~\ref{subsec (alu/exp): byte decompositions})
$\col{B\_0} \neq \byteCol{B\_0}_{i}$. With this in mind we further impose

\begin{enumerate}[resume]
	\item \If $\ct_{i} = \mmediumMO$ \Then
	\begin{enumerate}
		\item \If
		$\theta \cdot \Big(\col{B\_3} + \col{B\_2} + \col{B\_1}\Big)
		+ \col{B\_0}
		\neq
		\byteCol{B\_0}_{i}$
		\Then
		\[
		\left\{
		\begin{array}{lr}
			\displaystyle
			\If \byteCol{C\_0}_{i} = 1 ~ \Then
			8 \cdot \byteCol{H\_3}_{i} + \byteCol{H\_2}_{i} = 1 + \col{H\_1} \vspace{2mm}\\
			\displaystyle
			\If \byteCol{C\_0}_{i} = 0 ~ \Then
			8 \cdot \byteCol{H\_3}_{i} + \byteCol{H\_2}_{i} + \mmedium \cdot 8 = 1 + \col{H\_1} & (\trash) \\
		\end{array}
		\right.
		\]
		Note that the second constraint may be safely removed altogether as the left hand side is necessarily $\geq \mmedium \cdot 8 = 64$.
		\item \If
		$\theta \cdot \Big(\col{B\_3} + \col{B\_2} + \col{B\_1}\Big)
		+ \col{B\_0}
		=
		\byteCol{B\_0}_{i}$
		\Then
		\[
		\begin{cases}
			\If \byteCol{C\_0}_{i} = 1 ~ \Then
			\col{B\_0} \cdot \Big(8 \cdot \byteCol{H\_3} + \byteCol{H\_2} \Big)
			= 256 + \col{H\_1} \\
			\If \byteCol{C\_0}_{i} = 0 ~ \Then
			\col{B\_0} \cdot \Big(8 \cdot \byteCol{H\_3} + \byteCol{H\_2} + \mmedium \cdot 8 \Big)
			= 256 + \col{H\_1} \\
		\end{cases}
		\]
		\iffalse
		\item \If $\col{B\_3} + \col{B\_2} + \col{B\_1} = 0$ \Then
		\begin{enumerate}
			\item \If $\col{B\_0} \neq \byteCol{B\_0}_{i}$ \Then
			\[
			\begin{cases}
				\If \byteCol{C\_0}_{i} = 1 ~ \Then
				8 \cdot \byteCol{H\_3}_{i} + \byteCol{H\_2}_{i} = 1 + \col{H\_1} \\
				\If \byteCol{C\_0}_{i} = 0 ~ \Then
				8 \cdot \byteCol{H\_3}_{i} + \byteCol{H\_2}_{i} + \mmedium \cdot 8 = 1 + \col{H\_1} \\
			\end{cases}
			\]
			i.e. the lower bound on $2$-adicity that was extracted is nonzero;
			no further constraints are required since $\argTwo = \col{B\_0}$ is automatically $\geq 256$ by virtue of the two preceding hypotheses;
			furthermore the same remark as above applies here too.
			\item \If $\col{B\_0} = \byteCol{B\_0}_{i}$ \Then
			\[
			\begin{cases}
				\If \byteCol{C\_0}_{i} = 1 ~ \Then
				\col{B\_0} \cdot \Big(8 \cdot \byteCol{H\_3} + \byteCol{H\_2} \Big)
				= 256 + \col{H\_1} \\
				\If \byteCol{C\_0}_{i} = 0 ~ \Then
				\col{B\_0} \cdot \Big(8 \cdot \byteCol{H\_3} + \byteCol{H\_2} + \mmedium \cdot 8 \Big)
				= 256 + \col{H\_1} \\
			\end{cases}
			\]
		\end{enumerate}
		\fi
	\end{enumerate}
\end{enumerate}
The first set of constraints may \emph{altnernatively} be replaced with
\[
\begin{cases}
	\If \byteCol{C\_0}_{i} = 1 ~ \Then
	8 \cdot \byteCol{H\_3}_{i} + \byteCol{H\_2}_{i} \neq 0 \\
	\If \byteCol{C\_0}_{i} = 0 ~ \Then
	8 \cdot \byteCol{H\_3}_{i} + \byteCol{H\_2}_{i} + \mmedium \cdot 8 \neq 0 \\
\end{cases}
\]
these constraints are the true intent. They say that the lower bound on the 2-adicity (Again the second constraint may safely be removed for the same reasons.)

\iffalse
The parts about
``\ct_{i} = \mmediumMO ~ \Then
			8 \cdot \col{H\_2}
			+
			\col{H\_3} = 256 + \col{H\_1}''
are wrong. We need to multiply by the lowest byte etc ...
	\item
		\If $\byteCol{C\_0}_{i} = 1$\footnote{Given the above this is equivalent to $\byteCol{C\_0}_{i} \neq 0$, which one may choose instead for the implementation.} \Then
		\[
			\begin{cases}
				\prepareLowerBoundOnTwoAdicity
				\left(
				\begin{array}{c}
				\byteCol{A\_1},
				\byteCol{C\_1},
				\bits; \\
				\byteCol{C\_2},
				\byteCol{H\_2}; \\
				\byteCol{C\_3},
				\byteCol{H\_3};
				\ct{};
				\end{array}
				\right) \\
				\If \ct_{i} = \mmediumMO ~ \Then
				8 \cdot \col{H\_2}
				+
				\col{H\_3} = 256 + \col{H\_1}
			\end{cases}
		\]
		\item \If $\byteCol{C\_0}_{i} = 0$\footnote{Given the above this is equivalent to $\byteCol{C\_0}_{i} \neq 1$, which one may choose instead for the implementation.} \Then
		\[
			\begin{cases}
				\prepareLowerBoundOnTwoAdicity
				\left(
				\begin{array}{c}
				\byteCol{A\_1},
				\byteCol{C\_1},
				\bits; \\
				\byteCol{C\_2},
				\byteCol{H\_2}; \\
				\byteCol{C\_3},
				\byteCol{H\_3};
				\ct{};
				\end{array}
				\right) \\
				\If \ct_{i} = \mmediumMO ~ \Then
				8 \cdot \col{H\_2}
				+
				\col{H\_3} + \mmedium \cdot 8 = 256 +  \col{H\_1}
			\end{cases}
		\]
\fi
