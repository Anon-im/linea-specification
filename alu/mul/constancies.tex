As always, we say that a column $\col{X}$ is \textbf{stamp-constant} if it satisfies
\[
	\If \mulStamp_{i - 1} = \mulStamp_{i}
	~ \Then
	\col{X}_{i - 1} = \col{X}_{i}.
\]
We impose stamp-constancy conditions on the following columns:
\begin{multicols}{4}
\begin{itemize}
	\item $\argOneHi$;
	\item $\argOneLo$;
	\item $\argTwoHi$;
	\item $\argTwoLo$;
	\item $\resHi$;
	\item $\resLo$;
	\item $\INST$;
	\item[\vspace{\fill}]
\end{itemize}
\end{multicols}
\noindent Note that $\tinyExponent$, $\tinyBase$, $\oli$ and $\resVanishes$ are automatically stamp-constant seeing as they are directly derived from imported columns (see section~\ref{subsec (alu/exp): tiny base, tiny exponent, oli and result vanishes binary columns}.)

We also say that a column $\col{X}$ is \nBytes{}-constant if it satisfies
\[
	\Big(
	\modStamp_{i} = \modStamp_{i - 1}
	\et
	\nBytes_{i} = \nBytes_{i - 1}
	\Big)
	\implies
	\col{X}_{i} = \col{X}_{i - 1}.
\]
\ob{TODO: is \nBytes{}-constancy correctly defined ? \nBytes{} should be zero for \col{MUL} ? Do we even need it ? \dots{}}
