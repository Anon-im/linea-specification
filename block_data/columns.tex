We remind the reader that \ccc{} stands for ``counter-constant column.''
\begin{enumerate}
	\item $\ct$:
		counter column; hovers around zero and then cycles from $0$ to $\maxCtBtcData$;
	\item \blockNumberOfFirstBlockInConflation{}:
		``conflation-constant'' column containing the block number\footnote{In the sense of the \evm{}} of the first block of this conflation;
	\item $\relBlock$:
		\ccc{} containing the relative block number;
	\item $\relTxMax$:
		\ccc{} containing the number of transactions in this block;
	\item $\INST$:
		instruction column;
	\item $\blockDataHi$, $\blockDataLo$:
		columns containing batch data;
	\item $\coinbase\high$ and $\coinbase\low$:
		\ccc{} containing the
		coinbase address;
	\item \blockGasLimit{}:
		\ccc{} containing the
		block gas limit;
	\item \basefee{}:
		\ccc{} containing the
		base fee;
	\item $\byteCol{HI\_k}$, and $\byteCol{LO\_k}$, $k = 0, 1, \dots, \llargeMO$:
	\item $\wcpFlag$:
		binary flags used as selector for lookups;
\end{enumerate}
\saNote{}
The \INST{} column is $0$ during padding then cycles through a selection of ``block data'' specific opcodes e.g. \texttt{0x\,10} (i.e. \inst{COINBASE}), \texttt{0x\,46} (i.e. \inst{CHAINID}) etc\dots{} see section~(\ref{block data: value constraints}).
