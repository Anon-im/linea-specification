The \textbf{block data} module \btcMod{} is a repository for data which relates to the batch of transactions and which may find itself loaded into the stack. As such it serves data to the following instructions:
\begin{multicols}{3}
\begin{enumerate}
	%\item \inst{BLOCKHASH}
	\item \inst{COINBASE}
	\item \inst{TIMESTAMP}
	\item \inst{NUMBER}
	\item \inst{PREVRANDAO}
	\item \inst{GASLIMIT}
	\item \inst{CHAINID}
%	\item[\vspace{\fill}]
\end{enumerate}
\end{multicols}
Along with the \textsc{rom} and the \textsc{transaction data} module it serves as an entry point of outside data into the \zkEvm{}. As such one of its duties is to make sure the data it serves is correctly segmented (i.e. satisfies size constraints.)
