The present section describes the data layout for the \inst{CREATE} case. As such:
\[
	\boxed{\text{All constraints in this subsection further assume } \oobInstIsCreate_{i} = 1}
\]
We remind the reader that the objective of the \oobMod{} module in this particular case is to perform the following tasks.
First it detects aborting conditions for \inst{CREATE}-type instructions:
\green{(\emph{a})} compare \col{value} and \col{balance}, where \col{value} represents the transfer value (if any) of the instruction
\green{(\emph{b})} compare \col{callstackdepth} and $1024$.
Next it detects failure conditions for \inst{CREATE}-type instructions:
\green{(\emph{c})} the createe already has nonzero nonce
\green{(\emph{d})} the createe has nonempty byte code.

\noindent The interpretation of the \oobDataCol{k}'s is the following:
\[
	\left\{ \begin{array}{lclr}
		\locValueHi            & \define & \oobDataCol  {1}  _{i} \\
		\locValueLo            & \define & \oobDataCol  {2}  _{i} \\
		\locBalance            & \define & \oobDataCol  {3}  _{i} \\
		\locNonce              & \define & \oobDataCol  {4}  _{i} \\
		\locHasCode            & \define & \oobDataCol  {5}  _{i} \\
		\locCallStackDepth     & \define & \oobDataCol  {6}  _{i} \\
		\locAbortingCondition  & \define & \oobDataCol  {7}  _{i} & \prediction \\
		\locFailureCondition   & \define & \oobDataCol  {8}  _{i} & \prediction \\
	\end{array} \right.
\]
We impose the following constraints:
\begin{description}
	\item[\underline{Rows n°$(i)$, n°$(i + 1)$ and n°$(i + 2)$:}] 
		this row compares the balance to the value to transfer:
		\[
			\left\{ \begin{array}{l}
				\oobCallToLt
				{i}{0}
				{0}{\locBalance}
				{\locValueHi}{\locValueLo}
				\vspace{2mm} \\
				\oobCallToLt
				{i}{1}
				{0}{\locCallStackDepth}
				{0}{1024}
				\vspace{2mm} \\
				\oobCallToIszero
				{i}{2}
				{0}{\locNonce}
				\\
			\end{array} \right.
		\]
		and we introduce the following shorthands
		\[
			\left\{ \begin{array}{lcl}
				\locBalanceAbort  & \define & \outgoingResLo          _{i    } \\
				\locCsdAbort      & \define & 1 - \outgoingResLo      _{i + 1} \\
				\locNonzeroNonce  & \define & 1 - \outgoingResLo      _{i + 2} \\
			\end{array} \right.
		\]
		Thus by construction
		$\locBalanceAbort = 1 \iff \locBalance        <    \locValue $,
		$\locCsdAbort     = 1 \iff \locCallStackDepth \geq \col{1024}$ and
		$\locNonzeroNonce = 1 \iff \locNonce          \neq 0$.
	\item[\underline{Justifying \hubMod{} predictions:}] 
		we impose
		\[
			\left\{ \begin{array}{lcl}
				\locAbortingCondition & = &
				\left[ \begin{array}{crcl}
					+ & \locBalanceAbort       \\
					+ & (1 - \locBalanceAbort)  & \cdot & \locCsdAbort \\
				\end{array} \right] \vspace{2mm} \\
				\locFailureCondition  & = &
				(1 - \locAbortingCondition) \cdot
				\left[ \begin{array}{crcl}
					+ & \locHasCode       \\
					+ & (1 - \locHasCode)  & \cdot & \locNonzeroNonce \\
				\end{array} \right] \\
			\end{array} \right.	
		\]
		\saNote{} In other words
		$\locAbortingCondition$ tracks the \textbf{aborting conditions} for \inst{CREATE}'s instructions (which are the same as for \inst{CALL}-type instructions) while
		$\locFailureCondition$  tracks the \textbf{failure conditions} for \inst{CREATE}'s.

		\saNote{} By construction $\locAbortingCondition$ and $\locFailureCondition$ are \textbf{mutually exclusive binary flags}.
\end{description}

