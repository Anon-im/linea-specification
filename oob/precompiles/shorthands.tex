We will regularly, though not always, make use the following shorthands when dealing with precompiles.
\[
	\left\{ \begin{array}{lclr}
	       \locCalleeGas           & \define & \oobDataCol{1}_{i} \\
	       \locCds                 & \define & \oobDataCol{2}_{i} \\
	       \locRac                 & \define & \oobDataCol{3}_{i} \\
	       \locHubSuccess          & \define & \oobDataCol{4}_{i} & \quad \prediction \\
	       \locRamSuccess          & \define & \oobDataCol{4}_{i} & \quad \prediction \\
	       \locReturnGas           & \define & \oobDataCol{5}_{i} & \quad \prediction \\
	       \locExtractCallData     & \define & \oobDataCol{6}_{i} & \quad \prediction \\
	       \locEmptyCallData       & \define & \oobDataCol{7}_{i} & \quad \prediction \\
	       \locRacIsNonzero        & \define & \oobDataCol{8}_{i} & \quad \prediction \\
	\end{array} \right.
\]
\saNote{} The shorthands
\locHubSuccess{} and
\locRamSuccess{}
refer to \textbf{the same underlying cell} but at one of these names makes sense in any given context within the \oobMod{} module processing of any one precompile related \oobMod{}-instruction.

\saNote{} The decorator ``$\prediction$'' is used to label (\hubMod{} module) predictions.
