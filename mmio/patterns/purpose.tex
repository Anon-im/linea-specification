The present section compiles all variations on cutting, grafting and padding that the RAM needs and labels them. These \textbf{surgical patterns} are couched in a neutral setting in the sense that we use place holder names such as \source{} to \source{}\byte{}. These will later will be replaced with actual column names such as $\limb{}$ or $\byteA{}$. We also use \textbf{markers} for what will eventually be \textbf{byte offsets} $\in\{1,\dots,\llargeMO\}$.

We tend to use the same variable names over and over. Here is their general interpretation:
$(1)$ the letter \source{} and \target{} stand, respetively, for \col{source} and \col{target};
source and target limbs are assumed counter-constant;
source limbs are generally used as a source of bytes with which to modify one or more target limbs;
$(2)$ an exponent $(-)\new$ is meant to signal a ``new'' or ``updated'' value i.e. a value that is computed by the constraints; ``new'' values are always counter-constant;
$(3)$ the letter \byte{} stands for \col{byte};
$(4)$ the letter \mark{} stands for \col{marker} i.e. a ``byte marker'' or ``byte offset'' within a limb;
$(5)$ the letter \col{P} stands for \col{power}.
Thus the reader should interpret column names such as \sourceOne{}\mark{}, \targetTwo{}\byte{} and $\target\new$ as ``(byte) marker in the first source limb'',  ``bytes of the second target limb'' and ``new value of the target limb.''
Every surgical pattern is given a detailed interpretation before any constraints are written down. A picture accompanies it to make the intent clear.