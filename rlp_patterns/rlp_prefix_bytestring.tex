The present describes a contraint system that gives the RLP prefix of a bytestring. 
We shall assume that \textbf{the underlying byte string contains at least two bytes} (i.e. it's not empty and not composed of a single byte.) The other cases (empty byte string or single byte) are handled differently. We shall assume that we are provided with columns
\begin{itemize}
	\item $\col{length}$: integer containing the length of the bytestring; \emph{ought to be counter-constant};
	\item $\col{ct}$: a counter column;
	\item $\col{nStep}$: a constant column giving the size of the ct-loop;
	\item $\col{done}$: binary column that turns on a the end of every counter-cycle;
	\item $\col{isList}$: a binary column;
	\item $\col{byteSize}$: column counting the number of significant bytes of said input;
	\item $\col{power}$: a ``power of 256'' column;
	\item $\col{comp}$: a binary column used to compare the input with 55;
	\item $\col{acc1}$: an accumulator column that accumulates the bytes of an input;
	\item $\col{acc2}$: an accumulator column;
	\item $\col{lc}$: binary column that turns on the row where $\col{limb}$ is constructed;
	\item $\col{limb}$: column which will contain the prefix; 
	\item $\col{nBytes}$: number of meaningful bytes in the prefix itself;
\end{itemize}
In applications the interpretation will be as follows:
\col{acc1} will accumulate the bytes of an integer \col{length} representing the byte size of the underlying bytestring;
\col{isList} will distinguish between the two cases of interest: that of computing the \textsc{rlp} prefix of a simple byte string ($\in\mathbb{B}^*$) vs. of a list of items ($\in\mathbb{L} \setminus \mathbb{B}^*$);
the binary column \col{lc} will further satisfy a monotony condition (counter-incrementing) forcing it to remain $=1$ for the remainder of any counter-cycle as soon as it hits the value 1.

\saNote{}\label{rlp patterns: bytestring prefix: note: purpose of counter constancy of length} The only row on which the \col{length} column is constrained is the last one (i.e. any row with $\done \equiv 1$). The implied counter-constancy condition isn't all that important: in applications the value that one would like to use is the final one, but the \emph{correct} value to use might only be known at the start of the counter-cycle. In such cases one may use a counter-constant column, \emph{initialize} it at the start of the \ct{}-cycle with the desired value, and carry that value through to the end of the counter-cycle by means of counter-constancy.

\noindent We subsume under the shorthand
\[
	\rlpPrefixByteString_{i}
	\left(
	\begin{array}{r}
		\col{length},
		\col{ct},
		\col{nStep},
		\col{done},
		\col{isList}; \\
		\col{byteSize},
		\col{power},
		\col{comp},
		\col{acc1},
		\col{acc2}; \\
		\col{lc},
		\col{limb},
		\col{nBytes}; \\
	\end{array}
	\right)
\]
	
\begin{enumerate}
	\item $\rlpByteCounting_{i} (\col{ct}, \col{nStep}, \col{done}, \col{acc1}, \col{byteSize}, \col{power})$ 
	\item \If $\col{done}_{i} = 1$ \Then: 
		\begin{enumerate}
			\item $\col{acc1}_{i} = \col{length}_{i}$
			\item $\compareFiftyFive_{i}(\col{length}_{i},\col{comp}_{i},\col{acc2}_{i})$
			\item \If $\col{comp}_{i} = 0$ (case short input) \Then
				\begin{enumerate}
					\item $\col{lc}_{i - 1} + \col{lc}_{i} = 1$
					\item $\col{limb}_{i} = (\rlprefixShortInt \cdot (1 - \col{isList}) + \rlprefixShortList \cdot \col{isList}+\col{length}_{i})\cdot256^{\llargeMO}$
					\item $\col{nBytes}_{i} = 1$
				\end{enumerate}
				In the ``short input'' case the prefix occupies a single row containing (essentially) the \col{byteSize} itself.
			\item \If $\col{comp}_{i} = 1$ (case long input) \Then
				\begin{enumerate}
					\item $\col{lc}_{i - 2} + \col{lc}_{i - 1} = 1$
					\item $\col{limb}_{i - 1} = (\rlprefixLongInt \cdot (1 - \col{isList}) + \rlprefixLongList \cdot \col{isList}+\col{byteSize}_{i})\cdot256^{\llargeMO}$
					\item $\col{nBytes}_{i - 1} = 1$
					\item $\col{limb}_{i} = \col{length}_{i} \cdot \col{power}_{i}$
					\item $\col{nBytes}_{i} = \col{byteSize}_{i}$
				\end{enumerate} 
				In the ``long input'' case the prefix of the \textsc{rlp} encoding occupies two rows: the first one containing (essentially) the size in bytes of the length of the byte string (one byte suffices), the second containing (essentially) the byte size itself (which requires $\col{bytesize}\geq 1$ bytes)
		\end{enumerate}
\end{enumerate}
