The present describes a contraint system that writes the \rlp{}-prefix of certain data types given certain size constraints. The present pattern can be applied in the following situations:
(\emph{a})
obtaining the \rlp{}-prefix of (\emph{nonzero}) integers $n\in\mathbb{N}$ satisfying $0 < n < 256^{\llarge}$,
(\emph{b})
obtaining the \rlp{}-prefix of bytestrings of length $\leq \llarge$. \saNote{} Recall that the \rlp{} encoding of the integer $0\in\mathbb{N}$ is that of the empty bytestring $(\,) \in \mathbb{B}_0 \subset \mathbb{B}^*$, i.e. $\rlprefixShortInt$, while the \rlp{} encoding of the zero byte $\texttt{0x\,00}\in\mathbb{B}$ is itself.

We shall assume the following: we are provided with columns
\begin{itemize}
    \item $\col{input}$: an integer;
    \item $\col{ct}$: a counter;
    \item $\col{nStep}$: a column which distinguishes between the \emph{expected} maximal byte size;
    \item $\col{done}$: a binary columns which lights up at the end of the byte decomposition; \\
	\saNote{We have $\col{done}_{i}=1 \iff \col{ct}_{i}=\col{nStep}_{i}-1$.}
    \item $\col{limb}$: a column where the output is written;
    \item $\col{lc}$: a bit column, is 1 when something is written in the $\col{limb}$, else 0;
    \item $\col{nBytes}$: number of meaningfull bytes of the $\col{limb}$;
    \item $\col{byte}$: a byte column which is accumulated;
    \item $\col{acc}$: an accumulator column that accumulates the bytes of an input;
    \item $\col{byteSize}$: column counting the number of significant bytes of said input;
    \item $\col{power}$: a ``power of 256'' column;
    \item $\col{bit}$: a binary column used to do a bit decomposition;
    \item $\col{bitAcc}$: a bit accumulator.
\end{itemize}

\noindent We subsume under the short hand
\[
    \rlpPrefixInt_{i}
    \left(
	\begin{array}{r}
    \col{input},
    \col{ct},
    \col{nStep},
    \col{done}; \\
    \col{byte},
    \col{acc},
	\col{byteSize},
	\col{power},
	\col{bit},
	\col{bitAcc}; \\
    \col{limb},
    \col{lc},
    \col{nBytes}; \\
    \end{array}
	\right)
\]

\begin{enumerate}
    \item \rlpByteCounting (\col{ct}, \col{nStep}, \col{acc}, \col{byteSize}, \col{power}) 
    \item \If $\col{done}_{i}=1$ \Then: 
	\begin{enumerate}
	    \item $\col{acc}_{i}=\col{input}_{i}$
	    \item $\col{bitAcc}_{i}=\col{byte}_{i}$
	    \item \If ($\col{byteSize}_{i} = 1$ \et $\col{bit}_{i-7} = 0$) \Then $\col{lc}_{i-1}=0$
	    \item \If $\col{byteSize}_{i} \ne 1$ \Then $\texttt{rlpPrefixOfIntegerGeq128}$
	    \item \If $\col{bitAcc}_{i-7} \ne 0$ \Then $\texttt{rlpPrefixOfIntegerGeq128}$
	\end{enumerate}
\end{enumerate}

Where we use the following local shorthand:
\[
    \texttt{rlpPrefixOfIntegerGeq128} \iff
		\begin{cases}
		    \col{lc}_{i-2} + \col{lc}_{i-1}=1 \\
		    \col{limb}_{i-1}=(\rlprefixShortInt + \col{byteSize}_{i}) \cdot 256^{\llargeMO} \\
		    \col{nBytes}_{i-1}=1  \\
		\end{cases}
\]

\saNote{}The constraint $\col{lc}_{i-2} + \col{lc}_{i-1}=1 \iff \col{lc}_{i-2} = 0$ \et $\col{lc}_{i-1}=1$ as it will apply to a counter-incrementing column. 
\saNote{}The case \col{input}=0 is the integer 0 is not handled in this pattern.
\saNote{}In the case \col{input} < 128, there is nothing to write in the \col{limb} as there are no prefixs.

	
