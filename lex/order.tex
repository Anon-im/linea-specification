We introduce notations for lexicographic orders. Consider an integer $p \geq 2$ and a sign sequence $\epsilon_{\bullet} \in \{-, +\}^p$. Define the associated lexicographic order $\underset{\epsilon_{\bullet}}{\prec}$ on $p$-tuples nonnegative integers as follows. For \( \alpha_{\bullet} = (\alpha^1,\dots,\alpha^p),\, \beta_{\bullet} = (\beta^1,\dots,\beta^p) \in \mathbb{N}^p \) we set
\[
	\alpha_{\bullet} \underset{\epsilon_{\bullet}}{\prec} \beta_{\bullet}
	\iff
	\exists j \in \{1,\dots,p\},
	\begin{cases}
		\forall i \big( 1\leq i < j \implies \alpha_{i} = \beta_{i} \big) \\
		\epsilon_j(\beta_{i} - \alpha_{i}) > 0 \\
	\end{cases}
\]
For instance with $q=3$ and $\epsilon_{\bullet} = (+,+,-)$ the associated lexicographic order $\underset{\epsilon_{\bullet}}{\prec}$ satisfies
\[
	\alpha_{\bullet} \underset{\epsilon_{\bullet}}{\prec} \beta_{\bullet}
	\iff
	\begin{cases}
		& \alpha_{1} < \beta_{1} \\
		\OR & (\alpha_{1} = \beta_{1} \et \alpha_{2} < \beta_{2}) \\
		\OR & (\alpha_{1} = \beta_{1} \et \alpha_{2} = \beta_{2} \et \alpha_{3} > \beta_{3}) \\
	\end{cases}
\]

\saNote{} In the applications to come $\alpha_{\bullet}$ and $\beta_{\bullet}$ will typically be sampled from two consecutive rows of a pre-specified set of columns, i.e. there will be a set of columns $\ord^1, \ord^2, \dots, \ord^p$ and the relations that the constraints are meant to enforce are either a \textbf{strict ordering}:
\[
	\alpha_{\bullet} \underset{\epsilon_{\bullet}}{\prec} \beta_{\bullet}
\]
or a \textbf{weak ordering}
\[
	\alpha_{\bullet} \underset{\epsilon_{\bullet}}{\prec} \beta_{\bullet} ~\OR
	\alpha_{\bullet} = \beta_{\bullet}
\]
according to some pre-specified lexicographic order $\underset{\epsilon_{\bullet}}{\prec}$ where 
\[
	\begin{cases}
		\alpha_{\bullet} = \big[\ord_{i}^1, \ord_{i}^2, \dots, \ord^p_{i}\big] \vspace{2mm} \\
		\beta_{\bullet} = \big[\ord_{i + 1}^1, \ord_{i + 1}^2, \dots, \ord^p_{i + 1}\big] \\
	\end{cases}
\]
