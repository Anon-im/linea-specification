We list the named columns of the word comparison module. The first two dictate its (simple) heartbeat.
\begin{enumerate}
    \item $\wcpStamp$:
	column containing the word comparison stamp;
    \item \ct{}:
	counter column; 
	either hovers around $0$ or counts continuously up from $0$ to \maxCt{} whereupon it resets;
    \item \maxCt:
	counter-constant column;
	indicates the value at which \ct{} must reset;
\end{enumerate}
The following columns contain imported values related to the instruction at hand as seen in the hub module.
\begin{enumerate}[resume]
    \item $\INST{}$:
	\godGiven{}
	instruction column;
    \item $\argOneHi$, $\argOneLo$:
	\godGiven{}
	contain the high and low parts respectively of the first instruction argument;
    \item $\argTwoHi$, $\argTwoLo$:
	\godGiven{}
	contain the high and low parts respectively of the second instruction argument;
    \item $\res$:
	\godGiven{}
	contains the (bit) result;
    \item \isLt{}, \isGt{}, \isSlt{}, \isSgt{}, \isEq{}, \isIszero{}, \isGeq{}, \isLeq{}:
	mutually exclusive binary columns which light up precisely for the eponymous instruction;
    \item $\oli{}$:
	binary column; equals $1$ \emph{if and only if} $\INST\in\{\inst{EQ},\inst{ISZERO}\}$;
    \item $\vli{}$:
	binary column; equals $1$ \emph{if and only if} $\INST\in\{\inst{LT}, \inst{GT}, \inst{GEQ}, \inst{LEQ}, \inst{SLT}, \inst{SGT}\}$;
\end{enumerate}
\saNote{} As usual \oli{} is shorthand for \OLI{} and \vli{} is shorthand for \VLI{};
the \oli{} tag applies to instructions that can \emph{always} be dealt with in a single processing row;
the \vli{} tag applies to instructions that may require a variable number of processing rows.

\saNote{} The processing of variable length instructions can take anywhere from 1 to $\llarge$ rows.
\begin{enumerate}[resume]
    \item $\BITS$:
	binary column;
    \item $\negCol{1}$, $\negCol{2}$:
	counter-constant binary columns;
\end{enumerate}
$\negCol{1}$ and $\negCol{2}$ contain the sign bit of $\argOne$ and $\argTwo$ respectively.
The sign bits are required to deal with the signed comparison operations \inst{SLT} and \inst{SGT}.
To extract them one must provide the bit decomposition of the most significant byte of $\argOne$ and $\argTwo$ respectively: the purpose of the $\BITS$ column is to store those bits.

\saNote{} For unsigned instructions (\inst{LT}, \inst{GT}, \inst{ISZERO}, \inst{EQ} as well as \inst{LEQ} and \inst{GEQ}) both bits will be set to zero.
\begin{enumerate}[resume]
    \item $\byteCol{k}$ and $\acc{k}$ for $k = 1,\dots, 6$:
	byte and associated accumulator columns;
    \item $\bit{k}$ for $k = 1, \dots, 4$:
	counter-constant binary columns;
\end{enumerate}
