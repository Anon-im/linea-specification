\begin{enumerate}
\item $\gms$: generic module stamp;
\end{enumerate}
This is the same as the one from the associated data module. There may be discrepancies between the two stamps: instructions that for instance don't require touching RAM (say because they have a zero size parameter) may not trigger the associated data module. This applies for instance to hash functions but also logs with zero size parameter. 

The other columns that appear in generic info modules depend on the module at hand. They may include the following: 
\begin{enumerate}[resume]
\item $\size$: a size parameter;
\item $\addr$: an address parameter;
\item $\col{VAL}\high, \col{VAL}\low$: some value parameter, e.g. the output of a hash;
\item other parameters (e.g. log topics etc \dots).
\end{enumerate}
There may be more columns (e.g. columns signaling the nature of the data being recorded if several different interpretations are possible, such as differentiating inputs of \texttt{SHA2-256} from inputs to \texttt{RIPEMD-160}.)
