% TODO: split into different files
\[
    \boxed{\text{All shorthands introduced in this subsection should only be used given that }
        \begin{cases}
            \isModexpLog _{i} = 1 \\
            \isMacro     _{i} = 1 \\
        \end{cases}}
\]
\noindent We use the shorthands defined below:
\[
    \left\{ \begin{array}{lcl}
        \locRawLeadingWordHi       & \define & \expMacroData       {1}   _{i}     \\
        \locRawLeadingWordLo       & \define & \expMacroData       {2}   _{i}     \\
        \locCdsCutoff              & \define & \expMacroData       {3}   _{i}     \\
        \locEbsCutoff              & \define & \expMacroData       {4}   _{i}     \\
        \locLeadingWordLog         & \define & \expMacroData       {5}   _{i}     \\
        \locTrimAcc                & \define & \compTrimAcc              _{i - 1} \\
        \locNBytesExcludingLeading & \define & \compTanzbAcc             _{i - 2} \\
        \locNBitsExcludingLeading  & \define & \compManzbAcc             _{i - 2} \\
    \end{array} \right.
\]

% add shorthand for trim_lead and msnb potentially

We represent the desired data lay out in the table below:

\[
    % \hspace*{-1.5cm}
    \renewcommand\arraystretch{1.3}
\begin{array}{|c|c|c|c|c|c|c||c|} \hline
\locInc {1} & \locInc {2} & \locInc {3} & \locInc {4} & \locInc {5} & \locInc {6} & \oobDataCol{7}{1} & \locIncomingInstruction
           \\ \hline \hline
    \[
        \left\{ \begin{array}{l}
                \setExpInstructionParametersModexpLog {i}{\relof}
                \left[ \begin{array}{ll}
                        \utt{Raw leading word high:}     & \col{raw\_lead\_hi}          \\
                        \utt{Raw leading word low:}      & \col{raw\_lead\_lo}          \\
                        \utt{Call data offset cutoff:}   & \col{cds\_cutoff}            \\
                        \utt{Exponent byte size cutoff:} & \col{ebs\_cutoff}            \\
                \end{array} \right] \vspace{2mm} \\
                \qquad \define
                \left\{ \begin{array}{lclr}
                        \miscExpInstruction  _{i + \relof} & = & \expInstModexpLog \vspace{2mm} \\
                        \miscExpDataCol  {1} _{i + \relof} & = & \col{raw\_lead\_hi}         \\
                        \miscExpDataCol  {2} _{i + \relof} & = & \col{raw\_lead\_lo}         \\
                        \miscExpDataCol  {3} _{i + \relof} & = & \col{cds\_cutoff}           \\
                        \miscExpDataCol  {4} _{i + \relof} & = & \col{ebs\_cutoff}           \\
                        \miscExpDataCol  {5} _{i + \relof} & = & \col{lead\_log}              & \prediction \\
                \end{array} \right.
        \end{array} \right.
\]
       \\ \hline
    \end{array}
\]

We then define $\locMinCutoff \define \min{(\locCdsCutoff,\locEbsCutoff)}$.
