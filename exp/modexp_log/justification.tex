We now go on to justify the \hubMod{} module prediction of \locLeadingWordLog{}.
\begin{enumerate}
      \item \If $\locTrivialTrim = 1$ \Then
            \begin{enumerate}
                  \item \If $\locRawHiIsZero = 1$ \Then $ \locLeadingWordLog = 0 $
                  \item \If $\locRawHiIsZero = 0$ \Then
                        \begin{enumerate}
                              \item \If $\locEbsCutoffLEQSixteen = 1$  \Then
                                      $ \locLeadingWordLog = 0 $\footnote{and $\locTrim$ is extracted from $\locRawLeadingWordHi$}
                              \item \If $\locEbsCutoffLEQSixteen = 0$ \Then
                                    $ \locLeadingWordLog = 8 \cdot (\locEbsCutoff - \llarge) $\footnote{and $\locTrim$ is extracted from $\locRawLeadingWordHi$}
                        \end{enumerate}
            \end{enumerate}
      \item \If $\locTrivialTrim = 0$ \Then
            \begin{enumerate}
                  \item \If $\locEbsCutoffLEQSixteen = 1$ \Then
                        \[ \locLeadingWordLog = \locPaddedLogBaseTwo - 8 \cdot (\llarge - \locEbsCutoff)\footnote{and $\locTrim$ is extracted from $\locRawLeadingWordHi$} \]
                  \item \If $\locEbsCutoffLEQSixteen = 0$ \Then
                        \begin{enumerate}
                              \item \If $\locRawHiIsZero = 1$ \Then
                                    \[ \locLeadingWordLog = \locPaddedLogBaseTwo - 8 \cdot (\evmWordSize - \locEbsCutoff)\footnote{and $\locTrim$ is extracted from $\locRawLeadingWordLo$} \]
                              \item \If $\locRawHiIsZero = 0$ \Then
                                    \[ \locLeadingWordLog = \locPaddedLogBaseTwo + 8 \cdot (\locEbsCutoff - \llarge)\footnote{and $\locTrim$ is extracted from $\locRawLeadingWordHi$} \]
                        \end{enumerate}
            \end{enumerate}
\end{enumerate}
\saNote{}
Special care has to be taken for the following edge case: the underlying \evm{} word is of the form
\[
        \locRawLeadingWord
        \;\equiv
        \overbrace{
        \underbrace{
                \phantom{\bigg|}
                \utt{00}\;
		\utt{00}\;
		\utt{00}\;
		\utt{00}\;
		\utt{00}\;
		\utt{00}\;
		\utt{00}\;
		\utt{00}\;
		\utt{00}\;
		\utt{00}\;
		\utt{00}\;
		\utt{00}\;
		\utt{00}\;
		\utt{00}\;
		\utt{00}\;
		 \utt{01}
                \phantom{\bigg|}
        }_{\displaystyle \in \mathbb{B}_{\llarge}}
        \utt{??}\; \cdots \; \underline{\texttt{??}}}^{\displaystyle \in \mathbb{B}_{\evmWordSize}},
\]
the exponent byte size (cutoff) \locEbsCutoff{} is $> \llarge$ and $\locCdsCutoff$ is $\geq \llarge$. This is the only configuration wherein the high part gets trimmed and after trimming equals 1.
When $\locEbsCutoff > \llarge$ the \locLeadingWord{} is a singular $\utt{01}$ followed by $\locEbsCutoff - \llarge$ many bytes so that (the floor of) its logarithm base $2$ is $8 \cdot (\locEbsCutoff - \llarge)$.
This is the purpose of the subcase ``\If $\locTrivialTrim = 1$ \et $\locRawHiIsZero = 0$.''

\saNote{} Similar precautions are taken in the \shfMod{} module.
