In the end the \zkEvm's constraint system is ``flattened'' to a vast collection of constraints for processing by the prover. This step removes all ``direction'' and ``intention'' with which the constraint \constraintAst{} is imbued. This tells us that the formal verification of \evm{} semantics should take place at the \constraintAst{} level.

But it also opens a completely different opportunity for formal verification: formally verifying the \textbf{corset constraint flattening rulebook and implementation}. The main points of inquiry (though my faith in the rulebook and its implementation is strong) would be:
\begin{enumerate}
	\item correctly flattening nested constraints (e.g. no ``forgotten terms'');
	\item correctly dealing with special constraints:
	\begin{Verbatim}[commandchars=\\\{\}]
(if-zero CONDITION
             \textcolor{gray!75}{;; CONDITION == 0 case}
             ZERO_CASE      
             \textcolor{gray!75}{;; CONDITION != 0 case (optional)}
             NONZERO_CASE)
	\end{Verbatim}
	(and similarly for \texttt{(if-not-zero ...)} and \texttt{(if-eq-else ...)} constraints) in terms of producing the right auxiliary columns satisfying the right constraints;
\end{enumerate}
\alexV{} was mentioning that this could be achieved through testing. One implements the rulebook in \coq, uses it to compile down constraints to the target (\go) and compares the output with that generated by the \corset{} flattening. This can then be done on a large sample of constraints (e.g. the entire \zkEvm{} constraint system.)
