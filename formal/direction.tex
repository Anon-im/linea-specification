Cosntraints, when flattened, lose all direction which is present initially in the constraints, e.g. both constraints below
\begin{Verbatim}
	(if-not-zero A (vanishes! B))
	(if-not-zero B (vanishes! A))
\end{Verbatim}
compile down to the same equation:
\begin{Verbatim}
	A * B = 0
\end{Verbatim}
Yet the \textbf{intention} of the first constraint is to \textbf{set the contents} of the \texttt{B} column under certain circumstances where one assumes \texttt{A} already known (and in case of vanishing.) Data flows from cells that are understood to be \godGiven{} to others that aren't.

Other constraints implicitly 
