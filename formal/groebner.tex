There is the possibility to optimize the prover as follows:
(\emph{a})
compute the flattened constraint system $\mathcal{C}_\moduleName$ of a module $\moduleName$ with columns $\col{A}$, $\col{B}$, $\col{C}$, \dots{}
(\emph{b})
these are multivariate polynomials in as many columns as there are columns in the module
(\emph{c})
these multivariate polynomials define an ideal $\langle\mathcal{C}_\moduleName\rangle$ in some polynomial ring $\mathbb{F}[\col{A}, \col{B}, \col{C}, \dots]$ equipped with some ``shift operator'' $\sigma$; Groebner basis techniques could extract from them a Groebner basis which should represent a minimal set of constraints for the present module;
(\emph{d})
a valid module trace represents in some sense a ``parametrized solution'' to the above along a subset of the base field; the parametrization space being e.g. some group of roots of unity and the aforementioned ``shift operator'' should correspond under paramatrization to the ``argument translation by a fixed primitive root'' operator; 
(\emph{e})
not all minimal systems should have the same utility, and one can imagine a weight function being used to rank one minimal system agains another; this weight function should be correlated to actual prover costs.
