In this section we provide the graphical representation of the lookup constraints for \inst{CALL}-type instructions.
\begin{figure}
	\centering
	\[
		\hspace*{-1.5cm}
		\renewcommand{\arraystretch}{1.3}
		\begin{array}{|c||c|c|c|c|c|c|c|c||c|}
			\hline
			\multicolumn{10}{|c|}{\text{\inst{CALL} representation (for \oogxSH{} i.e. $\locOogx = \one$)}} \\ \hline
			\ct         & \argOneHi & \argOneLo                     & \graym{ \argTwoHi } & \argTwoLo                      & \exoInst       & \resLo                 & \wcpLookupFlag  & \divLookupFlag  &                                             \\ \hline 
			0           & 0         & \locGasActl                   & \graym{0}           & 0                              & \inst{LT}      & \zeroRes               & \one            & \zero           & \unoC \phantom{WW} \\ \hline 
			1           & \valHi    & \valLo                        & \graym{0}           & 0                              & \inst{ISZERO}  & \locZeroValue          & \locCctv        & \zero           & \unoC              \\ \hline 
			2           & 0         & \locGasActl                   & \graym{0}           & \locGasUpfrontCall             & \inst{LT}      & \locOogx ~ (=\one)     & \one            & \zero           & \unoC              \\ \hline 
		\end{array}
	\]
	%
	\[
		\hspace*{-1.5cm}
		\renewcommand{\arraystretch}{1.3}
		\begin{array}{|c||c|c|c|c|c|c|c|c||c|}
			\hline
			\multicolumn{10}{|c|}{\text{\inst{CALL} representation (no \oogxSH{} i.e. $\locOogx = \rZero$)}} \\ \hline
			\ct         & \argOneHi & \argOneLo                     & \graym{ \argTwoHi } & \argTwoLo                                & \exoInst       & \resLo                 & \wcpLookupFlag  & \divLookupFlag  &                                             \\ \hline 
			0           & 0         & \locGasActl                   & \graym{0}           & 0                                        & \inst{LT}      & \zeroRes               & \one            & \zero           & \unoC \phantom{WW} \\ \hline 
			1           & \valHi    & \valLo                        & \graym{0}           & 0                                        & \inst{ISZERO}  & \locZeroValue          & \locCctv        & \zero           & \unoC              \\ \hline 
			2           & 0         & \locGasActl                   & \graym{0}           & \locGasUpfrontCall                       & \inst{LT}      & \locOogx ~ (=\rZero)   & \one            & \zero           & \unoC              \\ \hline 
			3           & 0         & {\cellcolor{\ramCol}\locDiff} & \graym{0}           & 64                                       & \inst{DIV}     & \locOneSixtyFourth     & \zero           & \one            & \duoC               \\ \hline 
			4           & \gasHi    & \gasLo                        & \graym{0}           & {\cellcolor{\ramCol}\locLOfGasDiff}      & \inst{LT}      & \locGasCompBit         & \one            & \zero           & \duoC               \\ \hline 
		\end{array}
	\]
	\captionsetup{singlelinecheck=off}
	\caption[.]{We use the following shorthands 
	The interpretation of the orange zero $\zeroRes$ is that the first comparison is required to return \texttt{false}.
	When $\locOogx \equiv 0$ the \locOneSixtyFourth{}-cell contains $\locOSFth{}$.}
\end{figure}
