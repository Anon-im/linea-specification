\subsubsection{Constraints}
%%%%%%%%%%%%%%%%%%%%%%%%%%%


Suppose we are given a bit $\col{b} \in \{ 0, 1\}$.
We let $\callSP\big[\col{b}\big]_{i}$ stand for the following collection of constraints on rows $i$ and $i+1$.
The first lot of constraints pertains to the first row (i.e. $\ct_{i} = 0$):
\begin{description}
	\item[\underline{Stack Item $n^\circ\,1$:}] the first stack item contains what will become the call data offset:
	\[
	\left\{
	\begin{array}{lcl}
		\stackItemHeight{1}_{i}	& \!\!\! = \!\!\! & \stackHeight_{i} - 2 - \col{b} \\
		\stackItemPop{1}_{i}	& \!\!\! = \!\!\! & 1 \\
		\stackItemStamp{1}_{i}	& \!\!\! = \!\!\! & \hubTau \cdot \hubStamp_{i} + 3 \\
	\end{array}
	\right.
	\]
	\item[\underline{Stack Item $n^\circ\,2$:}] the second stack item contains what will become the call data size:
	\[
	\left\{
	\begin{array}{lcl}
		\stackItemHeight{2}_{i}	& \!\!\! = \!\!\! & \stackHeight_{i} - 3 - \col{b} \\
		\stackItemPop{2}_{i}	& \!\!\! = \!\!\! & 1 \\
		\stackItemStamp{2}_{i}	& \!\!\! = \!\!\! & \hubTau \cdot \hubStamp_{i} + 4 \\
	\end{array}
	\right.
	\]
	\item[\underline{Stack Item $n^\circ\,3$:}] the third stack item contais what will become the offset at which return data may be written into RAM:
	\[
	\left\{
	\begin{array}{lcl}
		\stackItemHeight{3}_{i}	& \!\!\! = \!\!\! & \stackHeight_{i} - 4 - \col{b} \\
		\stackItemPop{3}_{i}	& \!\!\! = \!\!\! & 1 \\
		\stackItemStamp{3}_{i}	& \!\!\! = \!\!\! & \hubTau \cdot \hubStamp_{i} + 5 \\
	\end{array}
	\right.
	\]
	\item[\underline{Stack Item $n^\circ\,4$:}] the fourth stack item contains what will become the size in bytes allocated by the call for writing return data directly into RAM:
	\[
	\left\{
	\begin{array}{lcl}
		\stackItemHeight{4}_{i}	& \!\!\! = \!\!\! & \stackHeight_{i} - 5 - \col{b} \\
		\stackItemPop{4}_{i}	& \!\!\! = \!\!\! & 1 \\
		\stackItemStamp{4}_{i}	& \!\!\! = \!\!\! & \hubTau \cdot \hubStamp_{i} + 6 \\
	\end{array}
	\right.
	\]
	\item[\underline{Height update:}] $\stackHeight{}\new{}_{i} = \stackHeight{}_{i} - 5 - \col{b}$ \quad (\sanityCheck);
\end{description}
The second lot of constraints pertains to the second row (i.e. $\ct_{i + 1} = 1$).
\begin{description}
	\item[\underline{Stack Item ``$n^\circ\,5$'':}] contains the gas argument of the \inst{CALL}-type instruction:
	\[
	\left\{
	\begin{array}{lcl}
		\stackItemHeight{1}_{i + 1}	& \!\!\! = \!\!\! & \stackHeight_{i + 1}  \\
		\stackItemPop{1}_{i + 1}	& \!\!\! = \!\!\! & 1 \\
		\stackItemStamp{1}_{i + 1}	& \!\!\! = \!\!\! & \hubTau \cdot \hubStamp_{i + 1} \\
	\end{array}
	\right.
	\]
	\item[\underline{Stack Item ``$n^\circ\,6$'':}] contains the address argument of the \inst{CALL}-type instruction:
	\[
	\left\{
	\begin{array}{lcl}
		\stackItemHeight{2}_{i + 1}	& \!\!\! = \!\!\! & \stackHeight_{i + 1} - 1 \\
		\stackItemPop{2}_{i + 1}	& \!\!\! = \!\!\! & 1 \\
		\stackItemStamp{2}_{i + 1}	& \!\!\! = \!\!\! & \hubTau \cdot \hubStamp_{i + 1} + 1 \\
	\end{array}
	\right.
	\]
	\item[\underline{Stack Item ``$n^\circ\,7$'':}] \emph{may} contain a value argument (used for both \inst{CALL} and \inst{CALLCODE})
	\[
	\left\{
	\begin{array}{lcl}
		\stackItemHeight{3}_{i + 1}	& \!\!\! = \!\!\! & (\stackHeight_{i + 1} - 2) \cdot \col{b} \\
		\stackItemPop{3}_{i + 1}	& \!\!\! = \!\!\! & \col{b} \\
		\stackItemStamp{3}_{i + 1}	& \!\!\! = \!\!\! & (\hubTau \cdot \hubStamp_{i + 1} + 2)\cdot \col{b} \\
	\end{array}
	\right.
	\]
	\item[\underline{Stack Item ``$n^\circ\,8$'':}] the fourth stack item contains the success bit
	\[
	\left\{
	\begin{array}{lcl}
		\stackItemHeight{4}_{i + 1}	& \!\!\! = \!\!\! & \stackHeight_{i + 1} - 5 - \col{b} \\
		\stackItemPop{4}_{i + 1}	& \!\!\! = \!\!\! & 0 \\
		\stackItemStamp{4}_{i + 1}	& \!\!\! = \!\!\! & \hubTau \cdot \hubStamp_{i + 1} + 7 \\
		\stackItemValHi{4}_{i + 1}	& \!\!\! = \!\!\! & 0 \\
		\multicolumn{3}{l}{\stackItemValLo{4}_{i + 1} \in \{ 0, 1 \} } \\
	\end{array}
	\right.
	\]
	\saNote{} The last constraints impose a ``bithood'' constraint on the stack item meant to hold the success bit.
\end{description}
Recall that \inst{CALL}-type instructions require either $\bm{6}$ or $\bm{7}$ stack arguments (and always produce a success bit as output.)
Whether a \inst{CALL}-type instructions requires 6 or 7 arguments can be read off of $\decFlag{1}$, see section~\ref{hub: instruction handling: Call}
In applications $\col{b}$ will be replaced with $\decFlag{1}_{i}$.


\subsubsection{Graphical representation}
%%%%%%%%%%%%%%%%%%%%%%%%%%%%%%%%%%%%%%%%


We represent the stack pattern when $\decFlag{1}=0$ is in figure~\ref{fig: call stack pattern flag1 = 0} and similarly for $\decFlag{1}=1$ see figure~\ref{fig: call stack pattern flag1 = 1}.
\begin{figure}[h!]
\[
	\tag{$\col{b} = 0$}
	\begin{array}{r}
		\begin{array}{|l|r|r|r|r|}
		\cline{2-5}
		\multicolumn{1}{c|}{(\ct_{i} = 0)} &
		\multicolumn{1}{c|}{\text{Stack}} &
		\multicolumn{1}{c|}{\text{Stack}} &
		\multicolumn{1}{c|}{\text{Stack}} &
		\multicolumn{1}{c|}{\text{Stack}} \\
		\multicolumn{1}{c|}{} &
		\multicolumn{1}{c|}{\text{Item 1}} &
		\multicolumn{1}{c|}{\text{Item 2}} &
		\multicolumn{1}{c|}{\text{Item 3}} &
		\multicolumn{1}{c|}{\text{Item 4}} \\
		\hline
		\stackItemHeight{k}_{i} &
		\col{h} - 2 & \col{h} - 3 & \col{h} - 4 & \col{h} - 5 \\
		\hline
		\stackItemValHi{k}_{i}/\stackItemValLo{k}_{i} & \cdo{} & \cds & \retAtOff & \retAtCap{} \\
		\hline
		\stackItemPop{k}_{i} & 1 & 1 & 1 & 1 \\
		\hline
		\stackItemStamp{k}_{i} & \hubTau \cdot \col{st} + 3 & \hubTau \cdot \col{st} + 4 & \hubTau \cdot \col{st} + 5 & \hubTau \cdot \col{st} + 6 \\
		\hline
		\end{array} \vspace{3mm} \\
		\begin{array}{|l|r|r|c|r|}
		\cline{2-5}
		\multicolumn{1}{c|}{(\ct_{i + 1} = 1)} &
		\multicolumn{1}{c|}{\text{Stack}} &
		\multicolumn{1}{c|}{\text{Stack}} &
		\multicolumn{1}{c|}{\text{Stack}} &
		\multicolumn{1}{c|}{\text{Stack}} \\
		\multicolumn{1}{c|}{} &
		\multicolumn{1}{c|}{\text{Item 1}} &
		\multicolumn{1}{c|}{\text{Item 2}} &
		\multicolumn{1}{c|}{\text{Item 3}} &
		\multicolumn{1}{c|}{\text{Item 4}} \\
		\hline
		\stackItemHeight{k}_{i + 1} & \col{h} & \col{h} - 1 & \graym{\varnothing} & \col{h} - 5 \\
		\hline 
		\stackItemValHi{k}_{i + 1}/\stackItemValLo{k}_{i + 1} & \col{gas} & \col{address} & \graym{\varnothing} & \col{success} \\
		\hline
		\stackItemPop{k}_{i + 1} & 1 & 1 & \graym{\varnothing} & 0 \\
		\hline
		\stackItemStamp{k}_{i + 1} & \hubTau \cdot \col{st} & \hubTau \cdot \col{st} + 1 & \graym{\varnothing} & \hubTau \cdot \col{st} + 7 \\
		\hline
		\end{array}
	\end{array}
\]
\label{fig: call stack pattern flag1 = 0}
\caption{%
Graphical representation of $\callSP\big[\col{b}\big]_{i}$ for $\col{b} = 0$.
In applications $\col{b} = \decFlag{1}_{i}$ so that the above applies to \inst{DELEGATECALL} and \inst{STATICCALL} instructions.
Recall that \cdo{}, \retAtOff{}, \cds{}, \retAtCap{} are short hand for \CDO{}, \RETATOFF{}, \CDS{}, \RETATCAP{} respectively.
We write $\col{h} := \stackHeight_{i}$ and $\col{st} := \hubStamp_{i}$.}
\end{figure}

\begin{figure}[h!]
\[
	\tag{$\col{b} = 1$}
	\begin{array}{r}
		\begin{array}{|l|r|r|r|r|}
		\cline{2-5}
		\multicolumn{1}{c|}{(\ct_{i} = 0)} &
		\multicolumn{1}{c|}{\text{Stack}} &
		\multicolumn{1}{c|}{\text{Stack}} &
		\multicolumn{1}{c|}{\text{Stack}} &
		\multicolumn{1}{c|}{\text{Stack}} \\
		\multicolumn{1}{c|}{} &
		\multicolumn{1}{c|}{\text{Item 1}} &
		\multicolumn{1}{c|}{\text{Item 2}} &
		\multicolumn{1}{c|}{\text{Item 3}} &
		\multicolumn{1}{c|}{\text{Item 4}} \\ \hline
		\stackItemHeight{k}_{i} & \col{h} - 3 & \col{h} - 4 & \col{h} - 5 & \col{h} - 6 \\
		\hline
		\stackItemValHi{k}_{i}/\stackItemValLo{k}_{i} & \cdo{} & \cds & \retAtOff & \retAtCap{} \\
		\hline
		\stackItemPop{k}_{i} & 1 & 1 & 1 & 1 \\
		\hline
		\stackItemStamp{k}_{i} & \hubTau \cdot \col{st} + 3 & \hubTau \cdot \col{st} + 4 & \hubTau \cdot \col{st} + 5 & \hubTau \cdot \col{st} + 6 \\
		\hline
		\end{array} \vspace{3mm} \\
		\begin{array}{|l|r|r|r|r|}
		\cline{2-5}
		\multicolumn{1}{c|}{(\ct_{i + 1} = 1)} &
		\multicolumn{1}{c|}{\text{Stack}} &
		\multicolumn{1}{c|}{\text{Stack}} &
		\multicolumn{1}{c|}{\text{Stack}} &
		\multicolumn{1}{c|}{\text{Stack}} \\
		\multicolumn{1}{c|}{} &
		\multicolumn{1}{c|}{\text{Item 1}} &
		\multicolumn{1}{c|}{\text{Item 2}} &
		\multicolumn{1}{c|}{\text{Item 3}} &
		\multicolumn{1}{c|}{\text{Item 4}} \\ \hline
		\stackItemHeight{k}_{i + 1} & \col{h} & \col{h} - 1 & \col{h} - 2 & \col{h} - 6 \\
		\hline 
		\stackItemValHi{k}_{i + 1}/\stackItemValLo{k}_{i + 1} & \col{gas} & \col{address} & \col{value} & \col{success} \\
		\hline
		\stackItemPop{k}_{i + 1} & 1 & 1 & 1 & 0 \\
		\hline
		\stackItemStamp{k}_{i + 1} & \hubTau \cdot \col{st} & \hubTau \cdot \col{st} + 1 & \hubTau \cdot \col{st} + 2 & \hubTau \cdot \col{st} + 7 \\
		\hline
		\end{array}
	\end{array}
\]
\label{fig: call stack pattern flag1 = 1}
\caption{%
Graphical representation of $\callSP\big[\col{b}\big]_{i}$ for $\col{b} = 1$.
In applications $\col{b} = \decFlag{1}_{i}$ so that the above will apply to \inst{CALL} and \inst{CALLCODE} instructions.
Recall that \cdo{}, \retAtOff{}, \cds{}, \retAtCap{} are short hand for \CDO{}, \RETATOFF{}, \CDS{}, \RETATCAP{} respectively.
We write $\col{h} := \stackHeight_{i}$ and $\col{st} := \hubStamp_{i}$.}
\end{figure}
