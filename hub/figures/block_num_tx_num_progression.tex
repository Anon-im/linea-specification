\begin{figure}
\centering
\[
\renewcommand{\arraystretch}{1.5}
\begin{array}{|c|c||c|c|c|c|}
	\hline
	\relativeBlockNumber & \txNum & ~ \col{tw} ~ & ~ \col{ti} ~ & ~ \col{te} ~ & ~ \col{tf} ~\\
	\hline\hline
	\vdots & \vdots & \vdots & \vdots & \vdots & \vdots \\
	0 & 0 & 0 & 0 & 0 & 0 \\
	0 & 0 & 0 & 0 & 0 & 0 \\
	\hline\hline
	1 & 1 & \dots & \dots & \dots & \dots \\
	1 & 1 & \dots & \dots & \dots & \dots \\
	1 & 1 & \dots & \dots & \dots & \dots \\
	\hline
	1 & 2 & \dots & \dots & \dots & \dots \\
	\hline
	1 & 3 & \dots & \dots & \dots & \dots \\
	1 & 3 & \dots & \dots & \dots & \dots \\
	\hline
	1 & 4 & \dots & \dots & \dots & \dots \\
	1 & 4 & \dots & \dots & \dots & \dots \\
	1 & 4 & \dots & \dots & \dots & \dots \\
	1 & 4 & \dots & \dots & \dots & \dots \\
	\hline
	1 & 5 & \dots & \dots & \dots & \dots \\
	\hline\hline
	2 & 1 & \dots & \dots & \dots & \dots \\
	\hline
	2 & 2 & \dots & \dots & \dots & \dots \\
	2 & 2 & \dots & \dots & \dots & \dots \\
	\hline\hline
	3 & 1 & \dots & \dots & \dots & \dots \\
	3 & 1 & \dots & \dots & \dots & \dots \\
	3 & 1 & \dots & \dots & \dots & \dots \\
	3 & 1 & \dots & \dots & \dots & \dots \\
	3 & 1 & \dots & \dots & \dots & \dots \\
	3 & 1 & \dots & \dots & \dots & \dots \\
	\hline\hline
	4 & 1 & \dots & \dots & \dots & \dots \\
	\hline
	4 & 2 & \dots & \dots & \dots & \dots \\
	\hline
	4 & 3 & \dots & \dots & \dots & \dots \\
	4 & 3 & \dots & \dots & \dots & \dots \\
	4 & 3 & \dots & \dots & \dots & \dots \\
	4 & 3 & \dots & \dots & \dots & \dots \\
	\vdots & \vdots & \vdots & \vdots & \vdots & \vdots \\
\end{array}
\]
\caption{Typical succession of $[ \relativeBlockNumber, \txNum ]$ pairs. The numbers of times a \emph{nonzero} pair $[\col{b}, \col{t}]$ appears is always $\geq \ob{TODO}$ according to constraints on the
$\txWarm$,
$\txInit$,
$\txExec$ and
$\txFinl$ (which we have abbreviated to
\col{tw},
\col{ti},
\col{te} and
\col{tf} respectively for succinctness.) The succession represented above doesn't respect this condition. It's purely there to illustrate the fact that the pairs $[ \relativeBlockNumber, \txNum ]$ are necessarily listed in the natural lexicographic order on $\mathbb{N}^2$. Changes in the $[ \col{b}, \col{t} ]$ pair are either
$[ \col{b}, \col{t} ]\rightsquigarrow [ \col{b}, \col{t} + 1]$ or
$[ \col{b}, \col{t} ]\rightsquigarrow [ \col{b} + 1, 1]$. }
\end{figure}
