The following scenarios are used to deal with \inst{CREATE}-type instructions. \textbf{All columns below are binary}.
\begin{enumerate}[resume]
		% CREATE scenarios:
	\item \scenCreateException:
		lights up precisely for \inst{CREATE}-type instructions that raise a (non \staticxSH{}) exception;
	\item \scenCreateAbort:
		lights up precisely for \inst{CREATE}-type instructions that don't raise an exception but are aborted;
	\item \scenCreateFCondWillRevert:
		lights up precisely for \inst{CREATE}-type instructions that don't raise an exception, aren't aborted but raise a failure condition;
		on top of that the current execution context will (eventually) revert; 
	\item \scenCreateFCondWontRevert:
		lights up precisely for \inst{CREATE}-type instructions that don't raise an exception, aren't aborted but raise a failure condition;
		on top of that the current execution context won't revert; 
\end{enumerate}
Let us say that a \inst{CREATE}-instruction
\textbf{gets executed}\label{hub: instruction handling: create: definition of 'get executed'}
if it is \textbf{unexceptional}, \textbf{isn't aborted} and \textbf{doesn't trigger a failure condition}. 
The following columns are triggered for \inst{CREATE}-instructions that get executed. These scenarios are self describing. 
\begin{enumerate}[resume]
	\item \scenCreateEmptyInitCodeWontRevert{}
	\item \scenCreateEmptyInitCodeWillRevert{}
	\item \scenCreateNonEmptyInitCodeFailureWontRevert{}
	\item \scenCreateNonEmptyInitCodeFailureWillRevert{}
	\item \scenCreateNonEmptyInitCodeSuccessWontRevert{}
	\item \scenCreateNonEmptyInitCodeSuccessWillRevert{}
\end{enumerate}

