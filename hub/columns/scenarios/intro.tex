\textbf{Scenario-rows} are characterized by $\peekScenario \equiv 1$ and columns pertaining to that perspective are prefixed with the following symbol: $\scenarioSignifier$. 
The purpose of scenario-rows is to simplify the processing of certain instructions, most notably
\inst{RETURN} instructions,
\inst{CALL}-type instructions, and
\inst{CREATE}-type instructions.
These instructions are complex and give rise to a plethora of possible processing scenarios.
These scenarios will be captured by means of binary flags of the present perspective.
Using scenario-rows the present arithmetization is able to convert \textbf{complex (and numerous) pre-conditions} triggering a particular execution path into the following:
(\emph{a})
a \textbf{single prediction} in the form of a scenario flag turning on
(\emph{b})
a \textbf{verification} that the requisite (former pre-)conditions are met
(\emph{c})
the adequate instruction processing over the next few rows.
The upshot of this approach is a considerable \textbf{degree reduction} in processing constraints.
Indeed, high degree ``condition prefixes'' become low degree ``scenario announcements'' followed by constraints justifying the particular scenario and enforcing the requisite conditions relating to execution. 
