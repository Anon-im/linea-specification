\begin{center}
	\boxed{\text{We abbreviate ``context-constant (binary) column'' to \cncc{}
	(resp. \cncbc{}.)}}
\end{center}
\textbf{Context-rows}\label{def: stack row} are characterized by $\peekContext \equiv 1$ and columns pertaining to that perspective are prefixed with the following symbol: $\contextSignifier$.
Note that \textbf{context data is immutable} (except for those which concerns return data, see below.)
\begin{enumerate}
	\item $\cnCn$:
		column containing the $\CN$ of the context beeing peeked into currently;
	\item  $\cnCsd$:
		\cncc{} containing the \CSD{};
	\item \cnIsRootContext{}:
		\cncc{} binary column; indicates whether the current context is the root context of a transaction; logically equivalent to $\big[ \cnCsd = 1 \big]$;
	\item $\cnStatic$:
		\cncbc{} indicating whether the present context is static;
\end{enumerate}
% The \textbf{context type} of an execution context is defined to be
% $\cnType = 0$ for message call contexts and
% $\cnType = 1$ for deployment contexts.
The following columns pertain to the \textbf{account} whose balance and storage are accessible in the current execution context.
\begin{enumerate}[resume]
	\item $\cnAccountAddress\high$, $\cnAccountAddress\low$:
		\cncc{} containing the high and low part of the address of the underlying account;
	\item $\cnAccDepNum$:
		\cncc{} containing the deployment number of $\cnAccountAddress$ in the present execution context;
\end{enumerate}
The following columns pertain to the account whose \textbf{bytecode} is being run in the current execution context.
\begin{enumerate}[resume]
	\item $\cnCodeAddress\high$, $\cnCodeAddress\low$:
		\cncc{} containing the high and low part of the address of the bytecode running in the current execution context;
	\item $\cnCodeDepNum$:
		\cncc{} containing the deployment number of $\cnCodeAddress$ in the present execution frame;
	\item $\cnCodeDepStatus$:
		\cncbc{} containing the deployment status of $\cnCodeAddress$ in the present execution frame;
	\item $\cnCodeCfi$:
		\cncc{} containing the \CFI{} of the bytecode running in the current execution frame;
\end{enumerate}
The following columns contains a partial reflection of the \textbf{caller} context. 
\begin{enumerate}[resume]
	\item $\cnCallerAddress\high$, $\cnCallerAddress\low$:
		\cncc{} containing the high and low part of the caller's address as returned by the \inst{CALLER} opcode;
	\item $\cnCallValue$:
		\cncc{} containing the value in Gwei provided in the message call / deployment operation which spawned the present context; 
\end{enumerate}
The following columns contain information pertaining to call data:
\begin{enumerate}[resume]
	\item $\cnCallDataContextNumber$:
		\cncc{} containing the caller context's context number;
	\item $\cnCdo$:
		\cncc{} containing the \CDO{};
	\item $\cnCds$:
		\cncc{} containing the \CDS{};
	\item $\cnRao$:
		\cncc{} containing the \RAO{};
		specified by the caller context at the moment the present context was spawned;
	\item $\cnRac$:
		\cncc{} containing the \RAC{};
		specified by the caller context at the moment the present context was spawned;
\end{enumerate}
\saNote{} \cnCdo{} and \cnCds{} vanish for deployment contexts.
\saNote{} \cnRao{} and \cnRac{} vanish for deployment contexts and the root context of message call transactions.

The following columns pertain to the returner context i.e. the most recent execution context spawned by the present context from which execution has returned. 
\begin{enumerate}[resume]
	\item $\cnUpdate$:
		binary column;
	\item $\cnReturner$:
		contains the context number of;
		% \item \ob{TODO: is this obsolete ?} $\cnReturnerIsPrecompile$:
		% binary column indicating whether the returner is a precompile;
	\item $\cnRdo$:
		column containing the \RDO{}; specified by \cnReturner{} at "re-entry";
	\item $\cnRds$:
		column containing the \RDS{}; specified by \cnReturner{} at "re-entry";
\end{enumerate}
As already stated: all context data is \textbf{immutable} except for ``return data'' related columns. This immutability isn't (readily) apparent in the time-ordered traces. Rather it has to be enforced in a reordered version of the execution trace which juxtaposes in consecutive rows all \textbf{context-rows} ordered by context number $\cnCn$ and by \hubStamp{}.
The binary flag $\cnUpdate$ indicates when the mutable context data $\cnReturner$, $\cnRdo$, $\cnRds$ are allowed to change.
Let us provide further details.
When an execution context resumes after a child context finishes executing it receives return data%
\footnote{Said return data may be empty e.g. after a successful deployment, if the child context ended on a \inst{STOP} or a \inst{SELFDESTRUCT}, if the child context ended on a \inst{RETURN}/\inst{REVERT} with empty data, if the child context terminated on an exception, when returning from a \inst{CALL} to a precompile which triggered an internal error, \dots} That data lives in the child context's RAM (whose \cn{} will be stored \cnReturner{}) starting at some offset (stored in \cnRdo{}) and comprising a number of bytes (stored in \cnRds{}.) Contrary to the offset / size parameters introduced previously these parameters need not be context-constant. They are updated every time execution of the present context resumes after a message call or deployment.

\saNote{} The above will hold true for \inst{CALL}'s to precompiled contracts, too. \inst{CALL}'s to these contracts \emph{do not} produce tangible execution contexts in the present arithmetization i.e. the \hubMod{} module never enters an execution context with $\cnCodeAddress \in \{ 1, 2, \dots, 9 \}$. Yet \textsc{ram} may have to undergo modifications. Typically (and along the happy path)
(\emph{a})
extraction of a slice of bytes from the current execution context's \textsc{ram} to be stored in some dedicated \col{DATA} module (e.g.
\ecDataMod{},
\hashDataMod{},
\romMod{},
\dots{})
(\emph{b})
producing a full copy of the result (extracted from a dedicated \col{INFO} module, e.g. \hashInfoMod{}) onto the \textsc{ram} of a ``fictitious execution context \col{fec}''
(\emph{c})
copying the relevant slice of bytes from the return data which was just inserted into \col{fec}'s \textsc{ram} into the current execution context's \textsc{ram}. The ``fictitious execution context \col{fec}'' will be granted a \cn{} deduced from the current \hubStamp{}. It will be stored as the \cnReturner{} of the current execution context, and various return-data-related fields of the current execution context will be set appropriately. 
