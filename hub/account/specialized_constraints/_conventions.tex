We use the following conventions/abbreviations:
\begin{description}
	\item[\qquad$\relof$:] 
		short for \texttt{relative offset};
	\item[\qquad$\reldo$:] 
		short for \texttt{relative ``doing'' offset};
	\item[\qquad$\reluo$:] 
		short for \texttt{relative ``undoing'' offset};
\end{description}
They are placeholder variables that are replaced with small nonnegative integers in applications.
In applications we will have
\[
	0 \leq \relof, \qquad 0\leq \reldo < \reluo
\]
small nonnegative integers. With instructions that take two parameters, typically written
\[
	\texttt{perspectiveNameUndoXxxUpdate}\big[\reluo, \reldo\big]_{i}
\]
the interpretation is that such constraints $\reluo$ppend a row at row index $i + \reluo$ that $\reldo$eprecates the action performed at row $i + \reldo$ on some \texttt{xxx} field belonging to some given perspective.
It will implicitly require that both rows in question belong to a particular perspective where said operations make sense.
