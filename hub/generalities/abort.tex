This section (partially) constrains the \abort{} and $\fCond$ flags. They exist purely to assist with the processing of \inst{CALL}-type and \inst{CREATE}-type instructions. These families of instructions share a unique property in that \textbf{their execution may be aborted despite them not raising an exception}. This is the case if any of the following conditions apply:
\begin{description}
	\item[\csdAbortSH{}:] the caller / \creator{} context has call stack depth 1024;
	\item[\balAbortSH{}:] the caller / \creator{} account has insufficient balance for the desired transfer;
\end{description}
(the latter applying only to \inst{CALL} and \inst{CALLCODE} instructions since \inst{DELEGATECALL} and \inst{STATICCALL} don't attempt to transfer value.) The purpose of the \abort{} flag is to signal when either of these \textbf{generic aborting conditions} applies.

If the instruction produces neither an exception nor is aborted for the above reasons, it may still not lead to no execution. There are indeed \textbf{pre-execution failure conditions} which are triggered early in the processing of $\Lambda$ or $\Theta$ ($\Xi$, really.) \inst{CALL}-type instructions may come to a premature end if the callee is a precompile and either it is provided with insufficient gas or a precompile specific failure condition is satisfied. \inst{CREATE}-type instructions may come to a premature end if the \createe{} address has nonzero nonce or nonempty code (\deadAbortSH{}.)
The purpose of the $\fCond$ is to signal when one of these \textbf{instruction specific failure conditions} is satisfied. The ones pertaining to \inst{CREATE(2)} can be justified in the present module. Those pertaining to exceptional behaviour for precompiles require a module of their own to justify \ob{TODO: I love the clarity though --- things are falling into place : )}

The following are generic constraints that enforce when \emph{not} to turn these flags on.
\begin{enumerate}
	\item (\trash) $\abort$ and $\fCond$ are $\hubStamp$-constant \ob{TODO: in ``constancies'' section};
	\item (\trash) $\abort$ and $\fCond$ are binary \ob{TODO: in ``binary constraints'' section};
	\item \If $\txExec_{i} = 0$ \Then
		\[
			\begin{cases}
				\abort_{i} = 0 \\
				\fCond_{i} = 0 \\
			\end{cases}
		\]
	\item \If $\xAhoy_{i} = 1$ \Then
		\[
			\begin{cases}
				\abort_{i} = 0 \\
				\fCond_{i} = 0 \\
			\end{cases}
		\]
	\item \If $\abort_{i} = 1$ \Then $\fCond_{i} = 0$;
		% \end{enumerate}
		% \saNote{} One may sum up these conditions as follows:
		% \[
		% \left\{
		% \begin{array}{rcl}
		% \big[ (1 - \txExec_{i}) + \xAhoy_{i} \big]				& \!\!\! \cdot \!\!\! & \abort_{i} = 0 \vspace{2mm} \\
		% \big[ (1 - \txExec_{i}) + \xAhoy_{i} + \abort_{i} \big]	& \!\!\! \cdot \!\!\! & \fCond_{i} = 0 \vspace{2mm} \\
		% \end{array}
		% \right.
		% \]
		% \begin{enumerate}[resume]
	\item \If $\peekStack_{i} = 1$ \et $\stackDecCallFlag_{i} + \stackDecCreateFlag_{i} = 0$\footnote{Along stack-rows the $\stackDecCallFlag$ and $\stackDecCreateFlag$ are mutually exclusive binary flags; thus condition $\stackDecCallFlag + \stackDecCreateFlag = 0$ is equivalent to $\stackDecCallFlag + \stackDecCreateFlag \neq 1$} \Then $\abort_{i} = 0$;
	\item \If $\peekStack_{i} = 1$ \et $\stackDecCreateFlag_{i} = 0$ \Then $\fCond_{i} = 0$;
\end{enumerate}

\saNote{} \label{note: xahoy + abort + fcond are mutually exclusive binary flags} By construction $\xAhoy$, $\abort$, $\fCond$ are mutually exclusive (\hubStamp{}-constant) binary flags.


\ob{TODO: for future notice: we will use the $\fCond$ flag in case of \inst{CALL}-type instructions to deal with precompile specific failure conditions.}
