The constraints below enforce the behavior described above.
\begin{enumerate}
        \item we impose hub-stamp-constancy on
		\begin{multicols}{2}
			\begin{enumerate}
				\item $\refund$
				\item $\refund\new$
			\end{enumerate}
		\end{multicols}
	\item \If $\txExec_{i} = 0$ \Then
		\[
			\left\{ \begin{array}{lcl}
				\refund_{i}     & = & 0 \\
				\refund\new_{i} & = & 0 \\
			\end{array} \right.
		\]
	\item \If $\txExec_{i} = 1$ \et $\hubStamp_{i} \neq \hubStamp_{i - 1}$ \Then $\refund_{i} = \refund\new_{i - 1}$
\end{enumerate}
The above is the transition constraint for gas refunds.

Recall that \textbf{along stack-rows} (i.e. whenever $\peekStack \equiv 1$) one has
\[
	\stackDecStoFlag \cdot \decFlag{2} \equiv 1  \iff \stackInst \equiv \inst{SSTORE} \text{, see section~(\ref{hub: instruction handling: sto});}
\]
\begin{enumerate}[resume]
	\item \If $\cnWillRev_{i} = 1$
		\Then $\refund\new_{i} = \refund_{i}$
	\item \If $\peekStack_{i} = 1$ \et $\stackDecStoFlag  _{i} \cdot \decFlag{2}_{i} = 0$
		\Then $\refund\new_{i} = \refund_{i}$
	\item \If $\cnWillRev_{i} = 0$ \et $\peekStack_{i} = 1$ \If $\stackDecStoFlag_{i} \cdot \decFlag{2}_{i}  = 1$ \Then \dots{} see section~(\ref{hub: instruction handling: sto})
\end{enumerate}
The above thus enforces that
(\emph{a}) execution context that will revert don't contribute to the refunds
(\emph{b}) only the \inst{SSTORE} opcode may change the refund counter.
Details as to what exactly this entails will be presented in the relevant section.
