The arithmetization requires, at various points during opcode processing, access to the knowledge of whether an instruction produces an exception or not.
Yet, the various exception flags for which we produced some preliminary constraints in section~(\ref{hub: generalities: exceptions: binarity and constancy}) belong to the stack perspective.
In other words they are only ``readily available'' along stack-rows i.e. rows with $\peekStack = 1$.
The purpose of the \xAhoy{} flag is to make a summary of that information available outside of stack rows for as long as required, i.e. along all rows with a given \hubStamp{}.

The constraints for \xAhoy{} are as follows:
\begin{enumerate}
	\item \xAhoy{} is binary
	\item \xAhoy{} is hub-stamp-constant
	\item \If $\txExec_{i} = 0$ \Then $\xAhoy_{i} = 0$
	\item \If $\peekStack_{i} = 1$ \Then $\xAhoy_{i} = \locExceptionFlagSum_{i}$
\end{enumerate}
\saNote{} Hub-stamp-constancy was already imposed in section~(\ref{hub: heartbeat: constancy conditions}).

\saNote{} The \xAhoy{} flag is a refinement of the \cmc{} flag which we define in section~(\ref{hub: generalities: context: cmc flag}) in the sense that whenever $\xAhoy = 1$ then $\cmc = 1$, too. 
