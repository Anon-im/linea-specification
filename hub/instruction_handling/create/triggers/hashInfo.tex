We impose the following:
\begin{enumerate}
	\item \If 
		\(
		\left[ \begin{array}{ll}
			+ & \scenCreateException              _{i} \\
			+ & \scenCreateAbort                  _{i} \\
			+ & \scenCreateExecutionEmptyInitCode _{i} \\
		\end{array} \right] = 1
		\)
		\Then
		\( \locTriggerHashInfo \define \gZero \)
	\item \If $\scenCreateFCond = 1$ \Then $\locTriggerHashInfo \define \locMxpMtntop \cdot \locIsCreateTwo$

		\saNote{} In other words
		\begin{enumerate}
			\item \If $\locSizeLo \neq 0$ \Then $\locTriggerHashInfo \define \locIsCreateTwo$ (\trash)
			\item \If $\locSizeLo =    0$ \Then $\locTriggerHashInfo \define \gZero$          (\trash)
		\end{enumerate}
	\item \If $\scenCreateExecutionNonEmptyInitCode = 1$ \Then $\locTriggerHashInfo \define \locIsCreateTwo$
\end{enumerate}
\saNote{}
One \emph{may} subsume the above into a single constraint
\[
	\locTriggerHashInfo
	\define
	\locIsCreateTwo
	\cdot
	\left[ \begin{array}{cr}
		+ & \scenCreateFCond \cdot \locMxpMtntop \\
		+ & \scenCreateExecutionNonEmptyInitCode \\
	\end{array} \right]
\]
\saNote{} The \hashInfoMod{} is called by \inst{CREATE2} instructions to compute the \textbf{initialization code hash};
this is required to compute the \textbf{deployment address};
this is necessary \emph{iff} the following conditions are met: 
(\emph{a}) the instruction is a \inst{CREATE2}
(\emph{b}) the instruction doesn't raise an exception
(\emph{c}) the instruction isn't aborted
(\emph{d}) the initialization code is nonempty;

\saNote{} Hashing the initialization code \textbf{is required} in case of a failure condition (as it requires us computing the deployment address to confirm that it has either a nonzero nonce or nonempty code).

\saNote{} The only difference between the \locTriggerMmu{} and \locTriggerHashInfo{} flags is that the \mmuMod{} module is \emph{always} triggered in case of a deployment of nonempty bytecode, while in similar circumstances the \hashInfoMod{} module is only triggered if the instruction is a \inst{CREATE2}. 
