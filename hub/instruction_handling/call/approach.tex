We explain our general appproach for the instruction processing of \inst{CALL}'s.
The first step in instruction processing is detecting exceptions / proving their absence.
Recall that \inst{CALL}-type instructions can trigger the following exceptions:
\begin{itemize}
	\item \suxSH{}
	\item \staticxSH{} (may only be triggered by \inst{CALL})
	\item \mxpxSH{}
	\item \oogxSH{}
\end{itemize}
The processing we present below, like any instruction processing, applies \emph{if and only if} no \suxSH{} was triggered (as witnessed by the $\stackSux + \stackSox \equiv 0$ precondition.)
The second step in instruction processing (of \emph{unexceptional} \inst{CALL}-type instructions) is detecting aborting conditions / proving their absence.
Recall that a \inst{CALL}-type instruction will be aborted for any of the following reasons
\begin{itemize}
	\item \balAbortSH{}
	\item \csdAbortSH{}
\end{itemize}
\inst{CALL}-type instructions that make it past these hurdles then undergo specialized processing depending on whether the \inst{CALL} targets an \textbf{externally owned account}, a \textbf{smart contract} or a \textbf{precompile}.
In our arithmetization
\begin{description}
	\item[\textsc{Externally owned accounts}]
		are characterized by having empty code ($\accHasCode \equiv 0$) and not being a precompile ($\accTrmIsPrecompile \equiv 0$);
	\item[\textsc{Smart contracts}]
		are characterized by having nonempty code ($\accHasCode \equiv 1$);
	\item[\textsc{Precompiles}]
		are characterized by raising the a precompile flag (i.e. $\accTrmIsPrecompile \equiv 1$).
\end{description}
Instruction processing beyond that point is either done in a single phase or subdivided two phases.
In more detail:
\begin{itemize}
	\item
		For \inst{CALL}'s to \textsc{externally owned accounts}' processing happens in one go (i.e. \textbf{one phase}) and culminates in the update of the present execution context's return data (which is reset to being empty $(\,)$).

		The final processing row is a \textbf{context-row}.
	\item For \inst{CALL}'s to \textsc{smart contracts}' processing, too, happens in one go (i.e. \textbf{one phase}) and culminates with:
		\begin{itemize}
			\item Backpropagating certain values pertaining to the success / failure of the child context;
			\item Initializating the execution context spawned by the \inst{CALL};
			\item Setting up the first row of the execution of the newly spawned execution environment;
		\end{itemize}
		See section~(\ref{hub: instruction handling: call: finishing: cases: smc}) for more details.
		The final processing row is a \textbf{context-row}.
	\item For \inst{CALL}'s to \textsc{precompiles}' processing is subdivided into \textbf{two phases}:
		\begin{enumerate}
		        \item
				The first phase is similar to the processing of \inst{CALL}'s to \textsc{eoa}'s and to \textsc{smc}'s and their processing is shared.
				However, rather than finishing on a context-row the final row of this phase is a \textbf{scenario-row}, where, among other things, we determine the relevant precompile scenario

				This first phase culminates with a (second) scenario-row where we set, for some positive (scenario-dependent) row offset $\exampleOffset$,
				\[
					\left\{ \begin{array}{lcl}
						\peekScenario_{i + \exampleOffset}        & = & 1 \\
						\scenSomePrecompile _{i + \exampleOffset} & = & 1 \\
						\multicolumn{3}{l}{\!\!\!\! \left. \begin{array}{lcl}
							\scenPrcSuccessWillRevert _{i + \exampleOffset} & = & \varepsilon    \\
							\scenPrcSuccessWontRevert _{i + \exampleOffset} & = & \varepsilon'   \\
							\scenPrcFailureKnownToHub _{i + \exampleOffset} & = & \varepsilon''  \\
							\scenPrcFailureKnownToRam _{i + \exampleOffset} & = & \varepsilon''' \\
						\end{array} \right \} \text{ precisely one is set to $1$.}}
					\end{array} \right.
				\]
			\item
				The second phase is entirely determined by the precompile address and the success/failure prediction made on the final row of the first phase.
		\end{enumerate}
\end{itemize}
In all cases the first phase deals with detecting exceptions, aborting conditions and updates to accounts (as well as undoing said upates in case of future rollbacks.)
the first phase The precise precompile address being called, as well as the success or failure of the relevant precompile, have 
