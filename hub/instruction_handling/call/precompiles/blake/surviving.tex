\begin{center}
	\boxed{%
		\text{The constraints presented below assume that }
		\left\{ \begin{array}{lcl}
			\peekScenario             _{i} & = & 1 \\
			\scenBlake                _{i} & = & 1 \\
			\scenPrcFailureKnownToHub _{i} & = & 0 \\
		\end{array} \right.
		}
\end{center}
We are thus considering the first row of the second phase of the \hubMod{}'s processing of a \inst{CALL} to the \inst{BLAKE2f} precompile.
We are further assuming that the \hubMod{} module didn't detect an exceptional \inst{CALL} to the \inst{BLAKE2f} precompile (i.e. $\cds = \locBlakeCallDataSize$.)
The call may still fail in \textsc{ram} (or succeed.)
We fix the behaviour of the next row, i.e. row $n^°(i + \prcBlakeSecondMiscRowOffset)$:
\begin{description}
	\item[\underline{\underline{Miscellaneous-row $n^°(i + \prcBlakeSecondMiscRowOffset)$:}}]
		we impose that
		\begin{description}
			\item[\underline{Setting lookup flags:}]
				we impose
				\[
					\weightedMiscFlagSum
					{i}{\prcBlakeSecondMiscRowOffset}
					=
					\left[ \begin{array}{crcl}
						+ & \miscMmuWeight  & \!\!\! \cdot \!\!\! & \locOobResultRamSuccess \\
						+ & \miscOobWeight \\
					\end{array} \right]
				\]
				in other words
				\[
					\left\{ \begin{array}{lclr}
						\miscExpFlag _{i + \prcBlakeSecondMiscRowOffset} & = & \gZero                                       & (\trash) \\
						\miscMmuFlag _{i + \prcBlakeSecondMiscRowOffset} & = & \undefinedStar \quad \locOobResultRamSuccess & (\trash) \\
						\miscMxpFlag _{i + \prcBlakeSecondMiscRowOffset} & = & \rZero                                       & (\trash) \\
						\miscOobFlag _{i + \prcBlakeSecondMiscRowOffset} & = & \one                                         & (\trash) \\
						\miscStpFlag _{i + \prcBlakeSecondMiscRowOffset} & = & \gZero                                       & (\trash) \\
					\end{array} \right.
				\]
				We are thus always calling the \oobMod{} and only calling the \mmuMod{} if the call to the \oobMod{} 

				\saNote{} The shorthand \locOobResultRamSuccess{} which we label with $\undefinedStar$ will be set below.
			\item[\underline{Setting \oobMod{} values and shorthands:}] 
				we impose the following constraints
				\[
					\setOobInstructionBlakeParams {
						anchorRow   = i                            ,
						relOffset   = \prcBlakeSecondMiscRowOffset ,
						calleeGas   = \locCalleeGas                ,
						blakeR      = \locBlakeR                   ,
						blakeF      = \locBlakeF                   ,
					}
					\label{hub: instruction handling: call: precompiles: blake: hub success call}
				\]
				We also define the following shorthands
				\[
					\left\{ \begin{array}{lcl} \label{hub: instruction handling: call: precompiles: blake: shorthands}
						% & \define & \miscOobDataCol{1}_{i + \prcBlakeSecondMiscRowOffset} \\
						% & \define & \miscOobDataCol{2}_{i + \prcBlakeSecondMiscRowOffset} \\
						% & \define & \miscOobDataCol{3}_{i + \prcBlakeSecondMiscRowOffset} \\
						\locOobResultRamSuccess      & \define & \miscOobDataCol{4}_{i + \prcBlakeSecondMiscRowOffset} \\
						\locOobResultReturnGas       & \define & \miscOobDataCol{5}_{i + \prcBlakeSecondMiscRowOffset} \\
						% \locOobResultExtractCallData & \define & \miscOobDataCol{6}_{i + \prcBlakeSecondMiscRowOffset} \\
						% \locOobResultEmptyCallData   & \define & \miscOobDataCol{7}_{i + \prcBlakeSecondMiscRowOffset} \\
						% \locOobResultNonzeroRac      & \define & \miscOobDataCol{8}_{i + \prcBlakeSecondMiscRowOffset} \\
					\end{array} \right.
				\]
				\saNote{} We have, by construction and by the computation in the \oobMod{} module, the following relations:
				\[
					\locOobResultRamSuccess \equiv \text{binary} 
				\]
			\item[\underline{Setting \mmuMod{} values:}]
				\If $\miscMmuFlag_{i + \prcBlakeSecondMiscRowOffset} = 1$ \Then
				\[
					\setMmuInstructionParametersRamToExoWithPadding {
						anchorRow       = i                                ,
						relOffset       = \prcBlakeSecondMiscRowOffset     ,
						sourceId        = \cn_{i}                          ,
						targetId        = 1 + \hubStamp_{i}                ,
						auxiliaryId     = \nothing                         ,
						sourceOffsetLo  = \locPrcCdo + \locBlakeRparamSize ,
						size            = \locBlakeHmtDataSize             ,
						referenceSize   = \locBlakeHmtDataSize             ,
						successBit      = \nothing                         ,
						exoSum          = \exoWeightBlakeModexp            ,
						phase           = \phaseBlakeData                  ,
					}
				\]
				\saNote{} Recall that the \inst{BLAKE2f} precompile fails unless $\cds \equiv \locBlakeCallDataSize$.
				Also $\locBlakeCallDataSize = \locBlakeRparamSize + \locBlakeHmtDataSize + \locBlakeFparamSize$, with
				the ``rounds'' parameter occupying the first $\locBlakeRparamSize$ bytes and
				the ``f''      parameter occupying the last  $\locBlakeFparamSize$ bytes, i.e. the final byte.
				The above step thus extracts the $\locBlakeHmtDataSize = 13 \cdot \llarge$ bytes in between contains 
				``\texttt{h}'',
				``\texttt{m}'',
				``\texttt{t\_max}'' and
				``\texttt{t\_min}''.
		\end{description}
	\item[\underline{Setting \scenPrcFailureKnownToRam{}:}] 
		we impose
		\[
			\left\{ \begin{array}{lclr}
				\scenPrcSuccess           _{i} & = & \locOobResultRamSuccess     \\
				\scenPrcFailureKnownToRam _{i} & = & 1 - \locOobResultRamSuccess  & (\trash) \\
			\end{array} \right.
		\]
		\saNote{}
		With this we have set all success / failure flags.

		\saNote{}
		It is enough to impose \textbf{one} of the preceding constraints.
	\item[\underline{Justifying return gas prediction:}]
		we impose
		\[
			\locPrcReturnGas
			=
			\scenPrcSuccess_{i}
			\cdot
			\locOobResultReturnGas.
		\]
\end{description}
