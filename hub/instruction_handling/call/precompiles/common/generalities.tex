\begin{center}
	\boxed{%
		\text{The constraints presented below assume that }
		\left\{ \begin{array}{lcl}
			\peekScenario         _{i} & = & 1 \\
			\scenPrecompileCommon _{i} & = & 1 \\
		\end{array} \right.
		}
\end{center}
We are thus assuming that the present row is the first of the second phase of dealing with the 
\inst{ECRECOVER},
\inst{SHA2-256},
\inst{RIPEMD-160},
\inst{ECADD},
\inst{ECMUL} or
\inst{ECPAIRING}
precompiles.
We know that the current scenario-row $i$ will be followed by a miscellaneous-row $(i + \prcCommonMiscRowOffset)$.
\begin{description}
	\item[\underline{Miscellaneous row $n^°(i + \prcCommonMiscRowOffset)$:}] we impose
		\begin{description}
			\item[\underline{Setting lookup flags:}]
				we impose that
				\[
					\weightedMiscFlagSum
					{i}{\prcCommonMiscRowOffset}
					=
					\left[ \begin{array}{clcl}
						+ & \miscMmuWeight & \cdot & \locOobResultExtractCallData \\
						+ & \miscOobWeight \\
					\end{array} \right]
				\]
				in other words
				\[
					\left\{ \begin{array}{lclr}
						\miscExpFlag _{i + \prcCommonMiscRowOffset} & = & \gZero                       & (\trash) \\
						\miscMmuFlag _{i + \prcCommonMiscRowOffset} & = & \locOobResultExtractCallData & (\trash) \\
						\miscMxpFlag _{i + \prcCommonMiscRowOffset} & = & \gZero                       & (\trash) \\
						\miscOobFlag _{i + \prcCommonMiscRowOffset} & = & \one                         & (\trash) \\
						\miscStpFlag _{i + \prcCommonMiscRowOffset} & = & \gZero                       & (\trash) \\
					\end{array} \right.
				\]
				\saNote{}
				The shorthand \locOobResultExtractCallData{} will be defined shortly.
				It is gleaned from one of the outputs of the call to the \oobMod{} module (which is systematically triggered given the above.)
			\item[\underline{Setting \oobMod{} instruction:}] 
				we populate the \oobMod{} columns:
				\[
					\setOobInstructionCommon {
						anchorRow          = i,
						relOffset          = \prcCommonMiscRowOffset,
						oobInstruction     = \locOobInst,
						calleeGas          = \locCalleeGas,
						callDataSize       = \locPrcCds,
						returnAtCapacity   = \locPrcRac,
					}
				\]
				where
				\[
					\locOobInst \define
					\left[ \begin{array}{crcl}
						+ & \oobInstEcrecover & \cdot & \scenEcrecover     _{i} \\ 
						+ & \oobInstShaTwo    & \cdot & \scenShaTwo        _{i} \\ 
						+ & \oobInstRipemd    & \cdot & \scenRipemd        _{i} \\ 
						+ & \oobInstIdentity  & \cdot & \scenIdentity      _{i} \\ 
						+ & \oobInstEcadd     & \cdot & \scenEcadd         _{i} \\ 
						+ & \oobInstEcmul     & \cdot & \scenEcmul         _{i} \\ 
						+ & \oobInstEcpairing & \cdot & \scenEcpairing     _{i} \\ 
					\end{array} \right]
				\]
			\item[\underline{Defining \oobMod{} shorthands:}] 
				We also define the following shorthands
				\[
					\left\{ \begin{array}{lcl} \label{hub: instruction handling: call: precompiles: common: shorthands}
						% & \define & \miscOobDataCol{1}_{i + \prcCommonMiscRowOffset} \\
						% & \define & \miscOobDataCol{2}_{i + \prcCommonMiscRowOffset} \\
						% & \define & \miscOobDataCol{3}_{i + \prcCommonMiscRowOffset} \\
						\locOobResultHubSuccess      & \define & \miscOobDataCol{4}_{i + \prcCommonMiscRowOffset} \\
						\locOobResultReturnGas       & \define & \miscOobDataCol{5}_{i + \prcCommonMiscRowOffset} \\
						\locOobResultExtractCallData & \define & \miscOobDataCol{6}_{i + \prcCommonMiscRowOffset} \\
						\locOobResultEmptyCallData   & \define & \miscOobDataCol{7}_{i + \prcCommonMiscRowOffset} \\
						\locOobResultNonzeroRac      & \define & \miscOobDataCol{8}_{i + \prcCommonMiscRowOffset} \\
					\end{array} \right.
				\]
				\saNote{} \label{hub: instruction handling: call: precompiles: common: generalities}
				We refer the reader to the relevant section in the \oobMod{} module, in particular section~(\ref{oob: populating: common precompiles}).

				\saNote{} We have, by construction and by the computation in the \oobMod{} module, the following relations:
				\[
					\left\{ \begin{array}{lclr}
						\locOobResultHubSuccess      & \equiv & \text{binary} & \quad (\trash)              \\
						\locOobResultExtractCallData & \equiv & \text{binary} & \quad (\trash)              \\
						\locOobResultEmptyCallData   & \equiv & \text{binary} & \quad (\trash) \vspace{2mm} \\
						\multicolumn{3}{l}{\locOobResultExtractCallData + \locOobResultEmptyCallData = \locOobResultHubSuccess} & (\trash) \\
					\end{array} \right.
				\]
			\item[\underline{Setting \mmuMod{} instruction:}]
				\If $\miscMmuFlag_{i + \prcCommonMiscRowOffset} = 1$ \Then we impose that
				\begin{enumerate}
					\item \If $\scenIdentity _{i} = 1$ \Then
						\[
							\setMmuInstructionParametersRamToRamSansPadding {
								anchorRow       = i                       ,
								relOffset       = \prcCommonMiscRowOffset ,
								sourceId        = \cn_{i}                 ,
								targetId        = 1 + \hubStamp_{i}       ,
								sourceOffsetLo  = \locPrcCdo              ,
								size            = \locPrcCds              ,
								referenceOffset = 0                       ,
								referenceSize   = \locPrcCds              ,
								}
						\]
					\item \If $\scenPrecompileCommonNotIdentity _{i} = 1$ \Then
						\[
							\setMmuInstructionParametersRamToExoWithPadding {
								anchorRow      = i                                   ,
								relOffset      = \prcCommonMiscRowOffset             ,
								sourceId       = \cn_{i}                             ,
								targetId       = 1 + \hubStamp_{i}                   ,
								auxiliaryId    = \nothing                            ,
								sourceOffsetLo = \locPrcCdo                          ,
								size           = \locPrcCds                          ,
								referenceSize  = \undefinedStar \quad \locMmuRefSize ,
								successBit     = \relevantValue                      ,
								exoSum         = \undefinedStar \quad \locMmuExoSum  ,
								phase          = \undefinedStar \quad \locMmuPhase   ,
								}
						\]
				\end{enumerate}
				where the shorthands marked with $\undefinedStar$ are as of yet undefined. We define them as follows:
				\[
					\locMmuRefSize \define
					\left[ \begin{array}{crcl}
						+ & \redm{128} & \cdot & \scenEcrecover _{i}  \\
						+ & \locPrcCds & \cdot & \scenShaTwo    _{i}  \\
						+ & \locPrcCds & \cdot & \scenRipemd    _{i}  \\
						+ & \redm{128} & \cdot & \scenEcadd     _{i}  \\
						+ & \redm{96}  & \cdot & \scenEcmul     _{i}  \\
						+ & \locPrcCds & \cdot & \scenEcpairing _{i}  \\
					\end{array} \right]
				\]
				and
				\[
					\locMmuExoSum \define
					\left[ \begin{array}{crcl}
						+ & \exoWeightEcdata & \cdot & \scenEcrecover _{i}  \\
						+ & \exoWeightRipSha & \cdot & \scenShaTwo    _{i}  \\
						+ & \exoWeightRipSha & \cdot & \scenRipemd    _{i}  \\
						% + & \gZero           & \cdot & \scenIdentity  _{i}  \\
						+ & \exoWeightEcdata & \cdot & \scenEcadd     _{i}  \\
						+ & \exoWeightEcdata & \cdot & \scenEcmul     _{i}  \\
						+ & \exoWeightEcdata & \cdot & \scenEcpairing _{i}  \\
					\end{array} \right]
				\]
				and
				\[
					\locMmuPhase \define
					\left[ \begin{array}{crcl}
						+ & \phaseEcrecoverData    & \cdot & \scenEcrecover     _{i}  \\
						+ & \phaseShaTwoData       & \cdot & \scenShaTwo        _{i}  \\
						+ & \phaseRipemdData       & \cdot & \scenRipemd        _{i}  \\
						% + & \gZero                 & \cdot & \scenIdentity      _{i}  \\
						+ & \phaseEcaddData        & \cdot & \scenEcadd         _{i}  \\
						+ & \phaseEcmulData        & \cdot & \scenEcmul         _{i}  \\
						+ & \phaseEcpairingData    & \cdot & \scenEcpairing     _{i}  \\
					\end{array} \right]
				\]
				\saNote{}
				In the implementation of the above \mmuMod{}-instruction setting one may drop the global precondition ``$\scenPrecompileCommon \equiv 1$.''
				Indeed both constraints impose more stringent preconditions.
			\item[\underline{Some shorthands for elliptic curve precompiles:}]
				we will use the following shorthands
				\begin{description}
					\item[\underline{\inst{ECRECOVER} specific:}]
						we set
						\[
							\left\{ \begin{array}{lcl}
								\locAddressRecoveryFailure & \define & 
								\left[ \begin{array}{clcl}
									+ & \locOobResultEmptyCallData   \\
									+ & \locOobResultExtractCallData  & \cdot & (1 - \miscMmuSuccessBit_{i + \prcCommonMiscRowOffset}) \\
								\end{array} \right] \\
								\locAddressRecoverySuccess & \define & \locOobResultExtractCallData \cdot \miscMmuSuccessBit_{i + \prcCommonMiscRowOffset} \\
							\end{array} \right.
						\]
						\saNote{} These bits will be used in the processing of \inst{ECRECOVER}.
						In that context (and given the precompile succedes)
						\begin{itemize}
							\item $\locAddressRecoveryFailure \equiv 1$ precisely when return data is empty ($\textbf{o} = ()$)
							\item $\locAddressRecoverySuccess \equiv 1$ precisely when recovery is successful and return data is nonempty ($\textbf{o} \in \mathbb{B}_{32}$.)
						\end{itemize}

						\saNote{} We have, by construction, and given that the precompile at hand is \inst{ECRECOVER}, the following relations:
						\[
							\left\{ \begin{array}{lclr}
								\locAddressRecoveryFailure   & \equiv & \text{binary} & \quad (\trash) \\
								\locAddressRecoverySuccess   & \equiv & \text{binary} & \quad (\trash) \vspace{2mm} \\
								\multicolumn{3}{l}{\locAddressRecoveryFailure + \locAddressRecoverySuccess = \locOobResultHubSuccess} & \quad (\trash) \\
							\end{array} \right.
						\]

						\saNote{} We finish by reminding that from the point of view of the \hubMod{} the bit $\miscMmuSuccessBit_{i + \prcCommonMiscRowOffset}$ is a \textbf{prediction}.
						This prediction will be borne out in the \ecDataMod{} module.
					\item[\underline{\inst{ECADD}, \inst{ECMUL} and \inst{ECPAIRING} specific:}]  we set
						\[
							\left\{ \begin{array}{lcl}
								\locMalFormedData  & \define & \locOobResultExtractCallData \cdot (1 - \miscMmuSuccessBit_{i + \prcCommonMiscRowOffset}) \\
								\locWellFormedData & \define & 
								\left[ \begin{array}{clcl}
									+ & \locOobResultEmptyCallData                                        \\
									+ & \locOobResultExtractCallData & \cdot & \miscMmuSuccessBit_{i + \prcCommonMiscRowOffset} \\
								\end{array} \right] \\
							\end{array} \right.
						\]
						\saNote{} These bits will be used in the processing of \inst{ECADD}, \inst{ECMUL} and \inst{ECPAIRING}.
						These precompiles have the property that \textbf{empty call data} is acceptable and produces the following outputs:
						\begin{center}
						\begin{tabular}{|ll|}
							\hline
							\underline{\inst{ECADD} case:}
							& $\textbf{o} =
							\utt{00}\,
							\utt{00}\, \cdots \,
							\utt{00} \in \mathbb{B}_{64}$ \\
							& representing the point at infinity; \\
							\underline{\inst{ECMUL} case:}
							& $\textbf{o} =
							\utt{00}\,
							\utt{00}\, \cdots \,
							\utt{00} \in \mathbb{B}_{64}$ \\
							& representing the point at infinity; \\
							\underline{\inst{ECPAIRING} case:}
							& $\textbf{o} =
							\utt{00}\,
							\utt{00}\, \cdots \,
							\utt{01} \in \mathbb{B}_{32}$ \\
							& representing $\rOne$. \\ \hline
						\end{tabular}
						\end{center}
						\saNote{} We have, by construction, the following relations:
						\[
							\left\{ \begin{array}{lclr}
								\locMalFormedData    & \equiv & \text{binary} & \quad (\trash) \\
								\locWellFormedData   & \equiv & \text{binary} & \quad (\trash) \vspace{2mm} \\
								\multicolumn{3}{l}{\locMalFormedData + \locWellFormedData = \locOobResultHubSuccess} & \quad (\trash) \\
							\end{array} \right.
						\]
				\end{description}
			\item[\underline{Some constraints involving the success bit:}]
				we may want to impose the following
				\[
					\miscMmuSuccessBit_{i + \prcCommonMiscRowOffset} \cdot 
					\left[ \begin{array}{cl}
						+ & \scenShaTwo        _{i}  \\
						+ & \scenRipemd        _{i}  \\
					\end{array} \right]
					= 0 \qquad (\trash)
				\]
				The above simply means that $\miscMmuSuccessBit_{i + \prcCommonMiscRowOffset}$ must vanish for the following precompiles:
				\inst{SHA2-256},
				\inst{RIPEMD-160}.

				\saNote{} We will come back to the interpretation of both
				\locMmuRecoverSuccess{} and
				\locMmuWellFormedData{}.
				As the names suggest \locMmuRecoverSuccess{} lets the \hubMod{} module know whether recovery of a public address was successful in case of a \inst{CALL} to \inst{ECRECOVER}.
				As the names suggest \locMmuWellFormedData{} lets the \hubMod{} module know whether the input data to \inst{ECADD}, \inst{ECMUL} or \inst{ECPAIRING} is well formed.
			\item[\underline{Justifying scenario success / failure predictions:}]
				\scenPrcSuccess{}, \scenPrcFailureKnownToHub{} and \scenPrcFailureKnownToRam{} are, by construction, exclusive binary columns;
				furthermore we previously imposed that
				\[
					\left[ \begin{array}{cr}
						+ & \scenPrcSuccess           _{i} \\
						+ & \scenPrcFailureKnownToHub _{i} \\
						+ & \scenPrcFailureKnownToRam _{i} \\
					\end{array} \right]
					= 1 \qquad (\trash)
				\]
				recall further that several precompiles have $\scenPrcFailureKnownToRam_{i} \equiv 0$ automatically;
				we impose the following:
				\[
					\scenPrcSuccess_{i} = 
					\left[ \begin{array}{crcl}
						+ & \locOobResultHubSuccess & \cdot &
						\left[ \begin{array}{cl}
							+ & \scenEcrecover   _{i} \\
							+ & \scenShaTwo      _{i} \\
							+ & \scenRipemd      _{i} \\
							+ & \scenIdentity    _{i} \\
						\end{array} \right] \vspace{2mm} \\
						+ & \locWellFormedData & \cdot &
						\left[ \begin{array}{cl}
							+ & \scenEcadd       _{i} \\
							+ & \scenEcmul       _{i} \\
							+ & \scenEcpairing   _{i} \\
						\end{array} \right] \\
					\end{array} \right]
				\]
				and
				\[
					\scenPrcFailureKnownToHub_{i} = 
					\left[ \begin{array}{crcl}
						+ & (1 - \locOobResultHubSuccess) & \cdot &
						\left[ \begin{array}{cl}
							+ & \scenEcrecover   _{i} \\
							+ & \scenShaTwo      _{i} \\
							+ & \scenRipemd      _{i} \\
							+ & \scenIdentity    _{i} \\
							+ & \scenEcadd       _{i} \\
							+ & \scenEcmul       _{i} \\
							+ & \scenEcpairing   _{i} \\
						\end{array} \right] \\
					\end{array} \right]
				\]
				and
				\[
					\hspace*{-1.5cm}
					\scenPrcFailureKnownToRam_{i} = 
					\left[ \begin{array}{crcl}
						+ & \gZero & \cdot &
						\left[ \begin{array}{cl}
							+ & \scenEcrecover   _{i} \\
							+ & \scenShaTwo      _{i} \\
							+ & \scenRipemd      _{i} \\
							+ & \scenIdentity    _{i} \\
						\end{array} \right] \vspace{2mm} \\
						+ & \locMalFormedData & \cdot &
						\left[ \begin{array}{cl}
							+ & \scenEcadd       _{i} \\
							+ & \scenEcmul       _{i} \\
							+ & \scenEcpairing   _{i} \\
						\end{array} \right] \\
					\end{array} \right]
				\]
				\saNote{} The above is somewhat redundant. We include the full characterizations of
				\scenPrcSuccess{},
				\scenPrcFailureKnownToRam{} and
				\scenPrcFailureKnownToRam{} for greater clarity.
			\item[\underline{Justifying return gas prediction:}]
				we impose
				\begin{enumerate}
				        \item \If $\scenPrcFailure_{i} = 1$ \Then $\locPrcReturnGas = 0$
				        \item \If $\scenPrcSuccess_{i} = 1$ \Then $\locPrcReturnGas = \locOobResultReturnGas$
				\end{enumerate}
				\saNote{} The above may be subsumed under
				\[
					\locPrcReturnGas
					=
					\scenPrcSuccess_{i}
					\cdot
					\locOobResultReturnGas
					\quad (\trash)
				\]
		\end{description}
	\end{description}
