\begin{center}
	\boxed{%
		\text{The constraints presented below are written under the assumption that } 
		\left\{ \begin{array}{lcl}
			\peekScenario      _{i + \callFirstScenarioRowOffset}    & = & 1 \\
			\scenCallNoPrc     _{i + \callFirstScenarioRowOffset}    & = & 1 \\
		\end{array} \right.
		}
\end{center}
In other words we consider all \inst{CALL}-type instructions that are either
(\emph{a}) exceptional,
(\emph{b}) unexceptional and aborted,
(\emph{c}) unexceptional, unaborted targeting an externally owned account or a smart contract.
The only case we \textbf{don't} deal with here is that of
(\emph{d}) unexceptional, unaborted \inst{CALL}-type instructions targeting a precompile.
This task will be taken up in \emph{in part} in
section~(\ref{hub: instruction handling: call: peeking flags and nRows for precompiles first half}).
\begin{description}
	\item[\underline{Setting \nonStackRows{} for non precompiles:}]
		we can set \nonStackRows{} once and for all (as long as we are not entering a precompile, that is):
		\[
			\nonStackRows_{i - \callFirstStackRowOffset}
			=
			\left[ \begin{array}{lrcl}
				+ & (\callStaticxUpdateParentContextRowOffset                  + 1) & \!\!\! \cdot \!\!\! & \locStaticx                                                      \\
				+ & (\callMxpxUpdateParentContextRowOffset                     + 1) & \!\!\! \cdot \!\!\! & \locMxpx                                                         \\
				+ & (\callOogxUpdateParentContextRowOffset                     + 1) & \!\!\! \cdot \!\!\! & \locOogx                                                         \\
				+ & (\callAbortWillRevertUpdateCurrentContextRowOffset         + 1) & \!\!\! \cdot \!\!\! & \scenCallAbortWillRevert      _{i + \callFirstScenarioRowOffset} \\
				+ & (\callAbortWontRevertUpdateCurrentContextRowOffset         + 1) & \!\!\! \cdot \!\!\! & \scenCallAbortWontRevert      _{i + \callFirstScenarioRowOffset} \\
				+ & (\callEoaWillRevertCallerContextRowOffset                  + 1) & \!\!\! \cdot \!\!\! & \scenCallEoaSuccessWillRevert _{i + \callFirstScenarioRowOffset} \\
				+ & (\callEoaWontRevertCallerContextRowOffset                  + 1) & \!\!\! \cdot \!\!\! & \scenCallEoaSuccessWontRevert _{i + \callFirstScenarioRowOffset} \\
				+ & (\callSmcFailureWillRevertInitializeCalleeContextRowOffset + 1) & \!\!\! \cdot \!\!\! & \scenCallSmcFailureWillRevert _{i + \callFirstScenarioRowOffset} \\
				+ & (\callSmcFailureWontRevertInitializeCalleeContextRowOffset + 1) & \!\!\! \cdot \!\!\! & \scenCallSmcSuccessWillRevert _{i + \callFirstScenarioRowOffset} \\
				+ & (\callSmcSuccessWillRevertInitializeCalleeContextRowOffset + 1) & \!\!\! \cdot \!\!\! & \scenCallSmcFailureWontRevert _{i + \callFirstScenarioRowOffset} \\
				+ & (\callSmcSuccessWontRevertInitializeCalleeContextRowOffset + 1) & \!\!\! \cdot \!\!\! & \scenCallSmcSuccessWontRevert _{i + \callFirstScenarioRowOffset} \\
			\end{array} \right]
		\]
	\item[\underline{Setting the peeking flags:}]
		we set the peeking flags once and for all (as long as we are not entering a precompile, that is):
		\[
			\hspace*{-3.8cm}
			\left[ \begin{array}{lrcl}
				+ \!\!\! & \peekingSumStaticx              & \!\!\! \cdot \!\!\! & \locStaticx                                                      \\
				+ \!\!\! & \peekingSumMxpx                 & \!\!\! \cdot \!\!\! & \locMxpx                                                         \\
				+ \!\!\! & \peekingSumOogx                 & \!\!\! \cdot \!\!\! & \locOogx                                                         \\
				+ \!\!\! & \peekingSumAbortWillRevert      & \!\!\! \cdot \!\!\! & \scenCallAbortWillRevert      _{i + \callFirstScenarioRowOffset} \\
				+ \!\!\! & \peekingSumAbortWontRevert      & \!\!\! \cdot \!\!\! & \scenCallAbortWontRevert      _{i + \callFirstScenarioRowOffset} \\
				+ \!\!\! & \peekingSumEoaWillRevert        & \!\!\! \cdot \!\!\! & \scenCallEoaSuccessWillRevert _{i + \callFirstScenarioRowOffset} \\
				+ \!\!\! & \peekingSumEoaWontRevert        & \!\!\! \cdot \!\!\! & \scenCallEoaSuccessWontRevert _{i + \callFirstScenarioRowOffset} \\
				+ \!\!\! & \peekingSumSmcFailureWillRevert & \!\!\! \cdot \!\!\! & \scenCallSmcFailureWillRevert _{i + \callFirstScenarioRowOffset} \\
				+ \!\!\! & \peekingSumSmcFailureWontRevert & \!\!\! \cdot \!\!\! & \scenCallSmcSuccessWillRevert _{i + \callFirstScenarioRowOffset} \\
				+ \!\!\! & \peekingSumSmcSuccessWillRevert & \!\!\! \cdot \!\!\! & \scenCallSmcFailureWontRevert _{i + \callFirstScenarioRowOffset} \\
				+ \!\!\! & \peekingSumSmcSuccessWontRevert & \!\!\! \cdot \!\!\! & \scenCallSmcSuccessWontRevert _{i + \callFirstScenarioRowOffset} \\
			\end{array} \right]
			=
			\nonStackRows_{i - \callFirstStackRowOffset}
		\]
\end{description}
Below are the concrete values of \nonStackRows{} depending on the \inst{CREATE}-type instruction scenario: 
\[
	\begin{array}{|l|l|c|} \hline
		\multicolumn{2}{|l|}{\textsc{Scenario}}                        & \nonStackRows{} \\ \hline \hline
		\scenCallException{} & \suxSH{}     & 1                                                               \\ \hline
		\scenCallException{} & \staticxSH{} & (\callStaticxUpdateParentContextRowOffset                        + 1) \\ \hline
		\scenCallException{} & \mxpxSH{}    & (\callMxpxUpdateParentContextRowOffset                           + 1) \\ \hline
		\scenCallException{} & \oogxSH{}    & (\callOogxUpdateParentContextRowOffset                           + 1) \\ \hline \hline
		\multicolumn{2}{|l|}{\scenCallAbortWillRevert}      & (\callAbortWillRevertUpdateCurrentContextRowOffset         + 1) \\ \hline \hline
		\multicolumn{2}{|l|}{\scenCallAbortWontRevert}      & (\callAbortWontRevertUpdateCurrentContextRowOffset         + 1) \\ \hline \hline
		\multicolumn{2}{|l|}{\scenCallEoaSuccessWillRevert} & (\callEoaWillRevertCallerContextRowOffset                  + 1) \\ \hline
		\multicolumn{2}{|l|}{\scenCallEoaSuccessWontRevert} & (\callEoaWontRevertCallerContextRowOffset                  + 1) \\ \hline \hline
		\multicolumn{2}{|l|}{\scenCallSmcFailureWillRevert} & (\callSmcFailureWillRevertInitializeCalleeContextRowOffset + 1) \\ \hline
		\multicolumn{2}{|l|}{\scenCallSmcFailureWontRevert} & (\callSmcFailureWontRevertInitializeCalleeContextRowOffset + 1) \\ \hline
		\multicolumn{2}{|l|}{\scenCallSmcSuccessWillRevert} & (\callSmcSuccessWillRevertInitializeCalleeContextRowOffset + 1) \\ \hline
		\multicolumn{2}{|l|}{\scenCallSmcSuccessWontRevert} & (\callSmcSuccessWontRevertInitializeCalleeContextRowOffset + 1) \\ \hline
	\end{array}
\]
\saNote{} The ubiquitous ``$+1$'' accounts for the fact that the \nonStackRows{} count must include the scenario row with index $i + \callFirstScenarioRowOffset$.
