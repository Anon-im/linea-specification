\begin{center}
	\boxed{%
		\text{The shorthands introduced below should only be used given that }
			\left\{ \begin{array}{lcl}
				\peekScenario     _{i}                & = & 1 \\
				\scenReturnSum    _{i}                & = & 1 \\
			\end{array} \right.}
\end{center}
We introduce some shorthands that will be used within the instruction processing of \inst{RETURN}'s.
\[
	\left\{ \begin{array}{lcl}
		\locInst     & \define & \stackInst          _{i - \locStackRowOffset} \\
		% \locMxpx   & \define & \stackMxpx          _{i - \locStackRowOffset} \\
		% \locOogx   & \define & \stackOogx          _{i - \locStackRowOffset} \\
		% \locMaxcsx & \define & \stackMaxcsx        _{i - \locStackRowOffset} \\
		% \locIcpx   & \define & \stackIcpx          _{i - \locStackRowOffset} \\
		\locOffsetHi & \define & \stackItemValHi{1}  _{i - \locStackRowOffset} \\
		\locOffsetLo & \define & \stackItemValLo{1}  _{i - \locStackRowOffset} \\
		\locSizeHi   & \define & \stackItemValHi{2}  _{i - \locStackRowOffset} \\
		\locSizeLo   & \define & \stackItemValLo{2}  _{i - \locStackRowOffset} \\
		% \locTriggerHashInfo & \define &  \\
		\multicolumn{3}{l}{\locHashInfoCodeHashHi ~ \define ~ \stackHashInfoValHi _{i - \locStackRowOffset}} \\
		\multicolumn{3}{l}{\locHashInfoCodeHashLo ~ \define ~ \stackHashInfoValLo _{i - \locStackRowOffset}} \\
	\end{array} \right.
	\quad\text{and}\quad
	\left\{ \begin{array}{lcl}
		\multicolumn{3}{l}{\locDeploymentAddressHi ~ \define ~ \cnCodeAddress\high _{i + \locCurrentContextRowOffset}} \\
		\multicolumn{3}{l}{\locDeploymentAddressLo ~ \define ~ \cnCodeAddress\low  _{i + \locCurrentContextRowOffset}} \\
		\locIsRoot                     & \define & \cnIsRootContext                   _{i + \locCurrentContextRowOffset}                  \\
		\locDeploys                    & \define & \cnCodeDepStatus                   _{i + \locCurrentContextRowOffset}                  \\
		\locRao                        & \define & \cnRao                             _{i + \locCurrentContextRowOffset}                  \\
		\locRac                        & \define & \cnRac                             _{i + \locCurrentContextRowOffset}                  \\
		\locMxpMxpGas                  & \define & \miscMxpGasMxp                     _{i + \locFirstMiscRowOffset}                       \\
		\locMxpMxpx                    & \define & \miscMxpMxpx                       _{i + \locFirstMiscRowOffset}                       \\
		\locMxpMtntop                  & \define & \miscMxpTypeFourInstMayTriggerMmu  _{i + \locFirstMiscRowOffset}                       \\
		\locMxpSizeOneNonzeroAndNoMxpx & \define & \miscMxpSizeOneNonzeroNoMxpx       _{i + \locFirstMiscRowOffset}                       \\
		\locDeploymentCfi              & \define & \accCfi                            _{i + \locNonemptyDeploymentFirstAccountRowOffset}  \\
	\end{array} \right.
\]
We further define
\[
	\left\{ \begin{array}{lcl}
		\locTypeSafeReturnDataOffset & \define & \locOffsetLo \cdot \locMxpSizeOneNonzeroAndNoMxpx \\
		\locTypeSafeReturnDataSize   & \define & \locSizeLo                                        \\
	\end{array} \right.
\]
\saNote{} \label{hub: instruction handling: halt: return: type safe offsets and sizes}
We briefly explain what makes the above shorthand(s) ``type safe'' and preferrable to the ``unfiltered'' version(s).
The main issue that these address is the possibility of \texttt{offset} to be enormous without triggering a \mxpxSH{} due to having zero \texttt{size}.
The above prevents us from providing the caller/creator context with the following combo: zero \texttt{return data size} and humouguous \texttt{return data offset}.
See section~(\ref{hub: misc: mxp: type4: purpose of checking nonzeroness of size 1}) for a (slightly) more detailed explanation.

\saNote{}
Given the specification of
\[
	\sizeOneNonzeroNoMxpx[]
\]
see chapter~(\ref{chap: mxp}),
there is really no reason to filter \locSizeLo{} in the same way we filter \locOffsetLo{}.
Thus the above just provides us with a ``meaningful alias'' to \locSizeLo{} to have similarly named shorthands for later.

\saNote{}
We remind the reader of the interpretation of the \miscMxpTypeFourInstMayTriggerMmu{} acronym:
\[
	\YNMO
\]
and that the \mxpMod{} module is responsible for computing this (binary) value, see chapter~(\ref{chap: mxp}).
We further remind the reader that $\miscMxpTypeFourInstMayTriggerMmu \equiv 1$ if and only if no \mxpxSH{} has occurred and the size parameter of the \mxpMod{} call is nonzero.

\saNote{}
We similarly remind the reader of the interpretation of the \miscMxpSizeOneNonzeroNoMxpx{} acronym:
\[
	\sizeOneNonzeroNoMxpx[]
\]
and that the \mxpMod{} module is responsible for computing this (binary) value, see chapter~(\ref{chap: mxp}).
We further remind the reader that $\miscMxpTypeFourInstMayTriggerMmu \equiv 1$ if and only if no \mxpxSH{} has occurred and the size parameter of the \mxpMod{} call is nonzero.
