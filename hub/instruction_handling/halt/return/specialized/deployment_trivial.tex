\begin{center}
	\boxed{%
		\text{The constraints presented below assume }
		\left\{ \begin{array}{lcl}
			\peekScenario              _{i} & = & 1 \\
			\scenReturnEmptyDeployment _{i} & = & 1 \\
		\end{array} \right.}
\end{center}
As per the above this section deals with recognizing and handling \textbf{unexceptional and trivial} \inst{RETURN} instructions executed in a deployment context.
\begin{description}
	\item[\underline{The first account row $n^°(i + \locEmptyDeploymentFirstAccountRowOffset)$:}]
		we impose the following
		\[
			\left\{ \begin{array}{lclr}
				\accAddressHi                                       _{i + \locEmptyDeploymentFirstAccountRowOffset}   & = & \locDeploymentAddressHi \\
				\accAddressLo                                       _{i + \locEmptyDeploymentFirstAccountRowOffset}   & = & \locDeploymentAddressLo \\
				\multicolumn{3}{l}{\accSameBalance                   {i} {\locEmptyDeploymentFirstAccountRowOffset}} \\
				\multicolumn{3}{l}{\accSameNonce                     {i} {\locEmptyDeploymentFirstAccountRowOffset}} \\
				\accCodesize\new                                    _{i + \locEmptyDeploymentFirstAccountRowOffset}   & = & \locSizeLo \\
				\accCodehashHi\new                                  _{i + \locEmptyDeploymentFirstAccountRowOffset}   & = & \emptyKeccakHi \\
				\accCodehashLo\new                                  _{i + \locEmptyDeploymentFirstAccountRowOffset}   & = & \emptyKeccakLo \\
				\multicolumn{3}{l}{\accSameDeploymentNumber          {i} {\locEmptyDeploymentFirstAccountRowOffset}}            \\
				\accDeploymentStatus\new                            _{i + \locEmptyDeploymentFirstAccountRowOffset}   & = & 0 \\
				\multicolumn{3}{l}{\accSameWarmth                    {i} {\locEmptyDeploymentFirstAccountRowOffset}} \\
				\multicolumn{3}{l}{\accSameMarkedForSelfdestructFlag {i} {\locEmptyDeploymentFirstAccountRowOffset}} \\
				\accRomLexFlag                                      _{i + \locEmptyDeploymentFirstAccountRowOffset}   & = & 0 \\
				\accTrmFlag                                         _{i + \locEmptyDeploymentFirstAccountRowOffset}   & = & 0 \\
				\multicolumn{3}{l}{
					\standardDomSubStamps {
						anchorRow        = i,
						relOffset        = \locEmptyDeploymentFirstAccountRowOffset,
						domOffset        = 0,
					}
				} \\
			\end{array} \right.
		\]
		\saNote{}
		Given the scenario ($\scenReturnEmptyDeployment \equiv 1$) we of course have $\locSizeLo = 0$.

		\saNote{}
		The following constraints aren't required \emph{per se} as they hold implicitly:
		\[
			\left\{ \begin{array}{lclr}
				\accCodehashHi                _{i + \locEmptyDeploymentFirstAccountRowOffset}   & = & \emptyKeccakHi & (\trash) \\
				\accCodehashLo                _{i + \locEmptyDeploymentFirstAccountRowOffset}   & = & \emptyKeccakLo & (\trash) \\
				\accDeploymentStatus          _{i + \locEmptyDeploymentFirstAccountRowOffset}   & = & 1 & (\trash) \\
				\multicolumn{3}{l}{\accIsntPrecompile {i}{\locEmptyDeploymentFirstAccountRowOffset}} & (\trash) \\
			\end{array} \right.
		\]
	\item[\underline{The second account row $n^°(i + \locEmptyDeploymentSecondAccountRowOffset)$:}]
		\If $\scenReturnFromDeploymentEmptyByteCodeWillRevert _{i} = 1$ \Then
		\[
			\left\{ \begin{array}{lclr}
				\multicolumn{3}{l}{\accSameAddr                            {i}{\locEmptyDeploymentSecondAccountRowOffset}{\locEmptyDeploymentFirstAccountRowOffset}}              \\
				\multicolumn{3}{l}{\accUndoBalanceUpdate                   {i}{\locEmptyDeploymentSecondAccountRowOffset}{\locEmptyDeploymentFirstAccountRowOffset}}              \\
				\multicolumn{3}{l}{\accUndoNonceUpdate                     {i}{\locEmptyDeploymentSecondAccountRowOffset}{\locEmptyDeploymentFirstAccountRowOffset}}              \\
				\multicolumn{3}{l}{\accUndoCodeUpdate                      {i}{\locEmptyDeploymentSecondAccountRowOffset}{\locEmptyDeploymentFirstAccountRowOffset}}              \\
				\multicolumn{3}{l}{\accUndoDeploymentStatusAndNumberUpdate {i}{\locEmptyDeploymentSecondAccountRowOffset}{\locEmptyDeploymentFirstAccountRowOffset}} \vspace{2mm} \\
				\multicolumn{3}{l}{\accUndoWarmthUpdate                    {i}{\locEmptyDeploymentSecondAccountRowOffset}{\locEmptyDeploymentFirstAccountRowOffset}}              \\
				\multicolumn{3}{l}{\accSameMarkedForSelfdestructFlag       {i}{\locEmptyDeploymentSecondAccountRowOffset}}                                                        \\
				\accRomLexFlag _{i + \locEmptyDeploymentSecondAccountRowOffset}   & = & 0 \\
				\accTrmFlag    _{i + \locEmptyDeploymentSecondAccountRowOffset}   & = & 0 \\
				\multicolumn{3}{l}{
					\revertDomSubStamps {
						anchorRow        = i,
						relOffset        = \locEmptyDeploymentSecondAccountRowOffset,
						subOffset        = 1,
					}
				} \\
				% \ob{TODO: The second argument \dots{} just a typo ?}
				% {i}{\locEmptyDeploymentSecondAccountRowOffset}{\locEmptyDeploymentFirstAccountRowOffset}} \\
			\end{array} \right.
		\]
	\item[\underline{Setting the caller's new return data:}]
		we impose that
		\begin{description}
			\item[\underline{The ``will revert'' case:}] 
				\If $\scenReturnFromDeploymentEmptyByteCodeWillRevert_{i}~=~1$ \Then
				we impose that
				\begin{enumerate}
					\item \If $\locIsRoot = 1$ \Then we impose
						\[
							\readContextData
							{i}{\locCallerContextRowOffsetEmptyDeploymentWillRevert}
							{\cn_{i}}
						\]
					\item \If $\locIsRoot = 0$ \Then we impose
						\[
							\executionProvidesEmptyReturnData
							{i}{\locCallerContextRowOffsetEmptyDeploymentWillRevert} 
						\]
				\end{enumerate}
			\item[\underline{The ``won't revert'' case:}] 
				\If $\scenReturnFromDeploymentEmptyByteCodeWontRevert_{i}~=~1$ \Then
				we impose that
				\begin{enumerate}
					\item \If $\locIsRoot = 1$ \Then we impose
						\[
							\readContextData
							{i}{\locCallerContextRowOffsetEmptyDeploymentWontRevert}
							{\cn_{i}}
						\]
					\item \If $\locIsRoot = 0$ \Then we impose
						\[
							\executionProvidesEmptyReturnData
							{i}{\locCallerContextRowOffsetEmptyDeploymentWontRevert} 
						\]
				\end{enumerate}
		\end{description}
\end{description}
