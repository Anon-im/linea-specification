The present module performs the \rlp{} \textbf{decodings} and \textbf{encodings} associated with processing a batch of \textsc{Ethereum} transactions; it thereby justifies the functional fields of the transactions that enter the \zkEvm{}.
Transactions enter the present module\footnote{and the \zkEvm{}!} through their ``full'' \rlp{} encodings, i.e. as a rather arbitrary looking string.
In order to makes sense of a transaction the \zkEvm{} must decode this string and verify the validity of the underlying transaction (e.g. verifying the signature but also verifying the sender has sufficient balance and the transaction nonce is correct.)
The present module takes care of the ``making sense of a transaction'' part of this.
It furthermore produces the message whose hash along with the signature are provided as inputs to \inst{ECRECOVER} to obtain the sender address.
We provide more details below.

Recall that there are two \rlp{} strings of interest associated with a transaction $T$.
There is the ``full'' \rlp{} encoding of a transaction, $\widetilde{L}_{\mathrm{T}}(T)$ as defined in the \cite{EYP}.
This contains all the fields of the transaction as well as a signature.
Then there is the \rlp{} string obtained by \rlp{} encoding the ``functional fields'' of a transaction, $\widetilde{L}_{\mathrm{X}}(T)$ as defined in the \cite{EYP}.
This defines the message whose hash is (supposedly) signed in the full transaction.

The present module thus does the following two things.
First, it \textbf{successfully} performs the \rlp{} \textbf{decoding} of some input string\footnote{The present module isn't equipped to deal with malformed transactions.}.
It verifies that this string is of the form $\widetilde{L}_{\mathrm{T}}(T)$ for some well-formed transaction $T$.
In the process it extracts the fields of that transaction (GasPrice, Value, \textbf{accessList}, etc\dots{}).
Any such ``extraction step'' comprises one of the various ``\textbf{phases}'' of the arithmetization presented here.
Note that this \textbf{does not} include the verification of the validity of these fields (e.g. value, gas price, nonce, \dots) against account data.
The associated checks are performed in the \txnDataMod{} and \hubMod{} modules.
Second, it performs the \rlp{} \textbf{encoding} of the relevant data to construct $\widetilde{L}_{\mathrm{X}}(T)$.

Since the layout of the fields of a transaction varies according to the \textbf{transaction type} our arithmetization distinguishes between
\textsc{legacy} transations (i.e. Type 0),
\textsc{access list} transations (i.e. Type 1) and
\textsc{eip-1559} transations (i.e. Type 2).
There are furthermore phases which deal with the complexities of the \rlp{} encoding itself.
These comprise justifying \textbf{sizes} (in bytes) as well \textbf{sizes of sizes} and various comparisons against protocol-specific constants.
Since most of the fields that make up a transction are shared among the ``full'' transaction \rlp{} and the \rlp{} of the submessage which was signed we are able to run decoding and encoding in parallel.

\subsection{Data extraction}
For a given transaction $T$, labelled in this module by its absolute transaction number
$\absTxNum$, the concatenation of the most significant 
$\limbsize$ bytes of the 
$\limb$ when
$\lc$ and
$\lt$ (resp.
$\lx$) is one, and ordered by growing 
$\indext$\footnote{starting at zero.} (resp. $\indexx$) gives $\widetilde{L}_{\mathrm{T}}(T)$ (resp. $\widetilde{L}_{\mathrm{X}}(T)$) of this transaction $T$.
\newline
For a given transaction $T$, labelled in this module by its absolute transaction number $\absTxNum$
the concatenation of the
$\limb$ when
$\phase{9}_{i}=1$ \et{}
$\isprefix_{i}=0$, indexed by
$\indexData$ gives the Data.
