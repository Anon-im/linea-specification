\begin{center}
	\boxed{\text{Throughout this section it is assumed that $\phaseAccessList_{i}=1$.}}
\end{center} 

We remind the reader that $\nbAddr$, $\nStorageKeysInAccessList$, $\nStorageKeysInAccessEntry$, $\Depth1$ and $\Depth2$ are counter-constant (see \ref{constancy_constraints}).
\newline
Constraints on the end of the phase:
\begin{enumerate}
	\item \If $\phasesize_{i} \ne 0$ \Then $\phasend_{i}=0$
	\item\label{constraint: end of access set phase} \If ($\phasesize_{i}=0$ \et $\done_{i}=1$) \Then $\phasend_{i}=1$.
\end{enumerate}
Constraint on $\outgoingDataSymb$ as $\nbAddr$ and $\nStorageKeysInAccessList$ are initialised at the begining of the phase:
\begin{enumerate}[resume]
	\item \If $\phaseAccessList_{i-1}=0$ \Then:
		\begin{enumerate}
			\item $\outgoingDataSymb\high_{i}=\nStorageKeysInAccessList_{i}$
			\item $\outgoingDataSymb\low_{i}=\nbAddr_{i}$
			\item $\isprefix_{i}=1$
			\item $\Depth1_{i} = 0$
			\item $\Depth2_{i} = 0$
			\item $\Input1_{i} = \phasesize_{i}$
			\item \If $\nbAddr_{i}=0$ \Then: 
				\begin{enumerate}
					\item $\nbstep_{i}=1$
					\item $\limb_{i}=\rlprefixShortList \cdot 256 ^{\llargeMO}$
					\item $\limbsize_{i}=1$
				\end{enumerate}
			\item \If $\nbAddr_{i} \ne 0$ \Then $\nbstep_{i}=8$
		\end{enumerate} 
\end{enumerate}
We first compute the RLP prefix and in the case of an empty list, end the phase:
\begin{enumerate}[resume]
	\item \If ($\Depth 1_{i}=0$ \et $\nbAddr_{i} \ne 0$) \Then:
		\begin{enumerate}
			\item 
				\[
					\rlpPrefixByteString
					\left( \begin{array}{r}
						\Input1,
						\ct,
						\nbstep,
						\done,
						\phaseAccessList{}; \\
						\accsize,
						\Power,
						\phaseAccessList,
						\acc1,
						\acc2; \\
						\lc,
						\limb,
						\limbsize; \\
					\end{array} \right)
				\]
			\item \If $\done_{i}=1$ \Then:
				\begin{enumerate} 
					\item $\isprefix_{i+1}=1$
					\item $\Depth1_{i+1}=1$
					\item $\Depth2_{i+1}=0$	
				\end{enumerate}
		\end{enumerate}
\end{enumerate}
We now deal with an access list item. We first compute the RLP prefix of an access tuple:
\begin{enumerate}[resume]
	\item \If ($\isprefix_{i}=1$ \et $\Depth1_{i}=1$ \et $\Depth2_{i}=0$) \footnote{It is implemented \If $\isprefix_{i} \cdot \Depth1_{i} - \Depth2_{i}=1$ \Then:} \Then:
		\begin{enumerate}
			\item $\Input1_{i} = \col{ACCESS\_TUPLE\_BYTESIZE}_{i}$
			\item $\nbstep_{i}=8$
			\item 
				\[
					\rlpPrefixByteString
					\left( \begin{array}{r}
						\Input1,
						\ct,
						\nbstep,
						\done,
						\phaseAccessList{}; \\
						\accsize,
						\Power,
						\phaseAccessList,
						\acc1,
						\acc2; \\
						\lc,
						\limb,
						\limbsize; \\
					\end{array} \right)
				\]
			\item \If $\done_{i}=1$ \Then:
				\begin{enumerate}
					\item $\isprefix_{i+1}=0$
					\item $\Depth1_{i+1}=1$
					\item $\Depth2_{i+1}=0$ 
				\end{enumerate}
		\end{enumerate}
\end{enumerate}
Then we compute the RLP of an address:
\begin{enumerate}[resume]
	\item \If ($\isprefix_{i}=0$ \et $\Depth1_{i}=1$ \et $\Depth2_{i}=0$) \footnote{It is implemented \If  $\Depth1_{i} - \isprefix_{i} - \Depth2_{i}=1$ \Then:} \Then:
		\begin{enumerate}
			\item
				\[
					\left\{
						\begin{array}{lcl}
							\Input1_{i} & \!\!\! = \!\!\! & \col{Addr}^{hi}_{i} \\
							\Input2_{i} & \!\!\! = \!\!\! & \col{Addr}^{lo}_{i} \\
							\nbstep & \!\!\! = \!\!\! & 16 \vspace{2mm} \\
							\multicolumn{3}{l}{\texttt{rlpAddressConstraints}_{i}(\Input1, \Input2, \oli, \ct)} \\
						\end{array}
						\right.
					\]
				\item \If $\done_{i}=1$ \Then:
					\begin{enumerate}
						\item $\isprefix_{i+1}=1$
						\item $\Depth1_{i+1}=1$
						\item $\Depth2_{i+1}=1$
					\end{enumerate}
		\end{enumerate}
\end{enumerate}
Then we compute the RLP prefix of the list of StorageKey in this access tuple:
\begin{enumerate}[resume]

	\item \If ($\isprefix_{i}=1$ \et $\Depth1_{i}=1$ \et $\Depth2_{i}=1$) \footnote{It is implemented \If $\isprefix_{i} \cdot \Depth1_{i} \cdot \Depth2_{i}=1$ \Then:} \Then:

		\begin{enumerate}
			\item \If $\nStorageKeysInAccessEntry_{i}=0$ \Then:
				\begin{enumerate}
					\item $\nbstep_{i}=1$
					\item $\limb_{i}=\rlprefixShortList \cdot 256^{\llargeMO}$
					\item $\limbsize_{i}=1$
				\end{enumerate}
			\item \If $\nStorageKeysInAccessEntry_{i} \ne 0$ \Then:  
			\item $\nbstep_{i}=8$
			\item $\Input1_{i} = 33\cdot\nStorageKeysInAccessEntry_{i} \footnote{As a storage key in 32 bytes long, RLP(StorageKey) is 33 bytes long.}$
			\item 
				\[
					\rlpPrefixByteString
					\left( \begin{array}{r}
						\Input1,
						\ct,
						\nbstep,
						\done,
						\phaseAccessList{}; \\
						\accsize,
						\Power,
						\phaseAccessList,
						\acc1,
						\acc2; \\
						\lc,
						\limb,
						\limbsize; \\
					\end{array} \right)
				\]
		\end{enumerate}
\end{enumerate}
Then, if relevant, we compute the RLP of a StorageKey:
\begin{enumerate}[resume]
	\item \If ($\isprefix_{i}=0$ \et $\Depth1_{i}=1$ \et $\Depth2_{i}=1$)\footnote{It is implemented \If $\Depth1_{i} \cdot \Depth2_{i} - \isprefix_{i}=1$ \Then:} \linebreak \Then $\texttt{rlpStorageKeyConstraints}_{i}(\Input1, \Input2, \ct)$
\end{enumerate}
Hereafter we give the condition on the step to do after computing the RLP prefix of the list of StorageKey, or the RLP of a StorageKey. It is either compute a RLP(StorageKey) if $\nStorageKeysInAccessEntry_{i} \ne 0$, a RLP(access tuple) if ($\nStorageKeysInAccessEntry_{i} = 0$ and $\nbAddr \ne 0$), ot the end of the phase if $\nbAddr_{i}=\nStorageKeysInAccessList_{i}=0$:
\begin{enumerate}[resume]
	\item \If ($\Depth2_{i}=1$ \et $\done_{i}=1$) \Then:
		\begin{enumerate}
			\item \If $\nStorageKeysInAccessEntry_{i} \ne 0$ \Then:
				\begin{enumerate}
					\item $\isprefix_{i+1}=0$  
					\item $\Depth1_{i+1}=1$ 
					\item $\Depth2_{i+1}=1$
				\end{enumerate}
			\item \If $\nStorageKeysInAccessEntry_{i}=0$ \Then:
				\begin{enumerate}
					\item $\col{ACCESS\_TUPLE\_BYTESIZE}_{i}=0$
					\item \If $\nbAddr_{i} \ne 0$ \Then:
						\begin{enumerate}
							\item $\isprefix_{i+1}=1$
							\item $\Depth1_{i+1}=1$
							\item $\Depth2_{i+1}=0$
						\end{enumerate}
					\item \trash We remind the reader of constraint~\ref{constraint: end of access set phase}. At this point, $\phasesize_{i}=0$ is equivalent to ($\nbAddr_{i} = 0$ and $\nStorageKeysInAccessList_{i}=0$), so the phase ends.
				\end{enumerate} 
		\end{enumerate}
\end{enumerate}
Here we define initialisation and decrementation of $\phasesize$, $\col{ACCESS\_TUPLE\_BYTESIZE}$, $\nbAddr$, $\nStorageKeysInAccessList$ and $\nStorageKeysInAccessEntry$.
\begin{enumerate}[resume]
	\item \If $\Depth1_{i}=0$ \Then $\phasesize_{i} = \phasesize_{i+1}$.
	\item \If $\Depth1_{i}=1$ \Then $\phasesize_{i} = \phasesize_{i-1} - \lc_{i} \cdot \limbsize_{i}$. 
\end{enumerate}
In other words $\phasesize$ is initialized at the begining of $\phaseAccessList{}$ and then decreased by the number of bytes in the $\limb$ when $\lc$ is nonzero.
\begin{enumerate}[resume]

	\item \If ($\Depth1_{i}=1$ \et ($\isprefix_{i}=0$ \Or $\Depth2_{i}=1$) \footnote{It is implemented \If $\isprefix_{i} \cdot (1 - \Depth2_{i})=0$ \Then} \Then $\col{ACCESS\_TUPLE\_BYTESIZE}_{i}=\col{ACCESS\_TUPLE\_BYTESIZE}_{i-1} - \lc_{i} \cdot \limbsize_{i}$

\end{enumerate}
In other words $\col{ACCESS\_TUPLE\_BYTESIZE}$ is initialized at the begining of an access tuple loop, with initialisation value only depending on the value of $\nStorageKeysInAccessEntry$ 
\footnote{The bytesize of an access tuple is 21 (20 for address + 1 for its RLP prefix) + RLP of the list of StorageKey (from 1 to 4 depending of the length of the bytesize of the list of StorageKeys) + 33 per StorageKeys (32 + 1 for its RLP prefix). The RLP prefix of the list of StorageKey is computed when $\isprefix = 1$, $\Depth 1 =1$ and $\Depth 2 = 1$ thus constraining the value of $\col{ACCESS\_TUPLE\_BYTESIZE}$.
\begin{enumerate}
	\item \If $\nStorageKeysInAccessEntry_{i}=0$ \Then $\col{ACCESS\_TUPLE\_BYTESIZE}_{i}=(1+20) + 1$
	\item \If $\nStorageKeysInAccessEntry_{i}=1$ \Then $\col{ACCESS\_TUPLE\_BYTESIZE}_{i}=(1+20) + 1 + (1+32)$
	\item \If $\nStorageKeysInAccessEntry_{i} \in [\![ 2 ; 7 ]\!] $ \Then $\col{ACCESS\_TUPLE\_BYTESIZE}_{i}= (1+20) + 2 + (1+32) \cdot \nStorageKeysInAccessEntry$
	\item \If $\nStorageKeysInAccessEntry_{i} \in [\![ 8 ; 1985  ]\!]$ \footnote{In this case the bytesize of $\nStorageKeysInAccessEntry$ RLP(StorageKey) is a two bytes integer.} \Then $\col{ACCESS\_TUPLE\_BYTESIZE}_{i}= (1+20) + 3 + (1+32) \cdot \nStorageKeysInAccessEntry$
	\item \If $\nStorageKeysInAccessEntry_{i} \in [\![ 1986 ; 508400]\!]$ \Then $\col{ACCESS\_TUPLE\_BYTESIZE}_{i}= (1+20) + 4 + (1+32) \cdot \nStorageKeysInAccessEntry$
\end{enumerate}
}
and decreased by the number of bytes in the $\limb$ when $\lc$ is nonzero during the loop (except during the access tuple RLP prefix).
\newline
For the following, we remind the reader that $\nbAddr$, $\nStorageKeysInAccessList$ and $\nStorageKeysInAccessEntry$ are $\ct$-constant (see \ref{constancy_constraints}).
\begin{enumerate}[resume]
	\item \If ($\Depth1_{i}=1$ \et $\ct_{i}=0$) \Then $\nbAddr_{i} = \nbAddr_{i-1} - \isprefix_{i} \cdot (1 - \Depth2_{i})$
\end{enumerate}
In other words, $\nbAddr$ is initialized at the begining of the phase, and then decrease by one at the begining of each RLP(access tuple).
\begin{enumerate}[resume]    
	\item \If ($\Depth1_{i}=1$ \et $\ct_{i}=0$) \Then $\nStorageKeysInAccessList_{i} = \nStorageKeysInAccessList_{i-1} - (1- \isprefix_{i}) \cdot \Depth2_{i}$
\end{enumerate}
In other words, $\nStorageKeysInAccessList$ is initialized at the begining of the phase, and then decrease by one at the begining of each RLP(Sto).
\begin{enumerate}[resume]
	\item \If (($\isprefix_{i}=0$ \Or $\Depth2_{i}=1$) \et $\ct_{i}=0$) \footnote{It is implemented \If $\isprefix_{i} \cdot (\Depth2_{i}-1) +\ct_{i} =0$ \Then} \Then $\nStorageKeysInAccessEntry_{i} = \nStorageKeysInAccessEntry_{i-1} - (1- \isprefix_{i}) \cdot \Depth2_{i})$
\end{enumerate}
In other words$\nStorageKeysInAccessEntry$ is initialized at the begining of an access tuple loop, and then decrease by one at the begining of each RLP(Sto).
\begin{enumerate}[resume]
	\item \If ($\Depth1_{i}=1$ \et $\nbAddr_{i} \neq \nbAddr_{i-1} -1$) \Then 
		\begin{enumerate}
			\item $\col{Addr}\high_{i} = \col{Addr}\high_{i-1}$
			\item $\col{Addr}\low_{i} = \col{Addr}\low_{i-1}$
		\end{enumerate}
\end{enumerate}
In other words $\col{Addr}\high$ and $\col{Addr}\low$ are constant all along an AccessTuple.
\saNote{} In our case, $\nbAddr_{i} \neq \nbAddr_{i-1} -1$ is equivalent to $\nbAddr_{i} = \nbAddr_{i-1}$

