\begin{enumerate}
	\item \blkMdxStamp{}:
		module stamp column;
	\item \modexpBlakeId{}:
		unique data identifier;
		can have large increments but must be listed in ascending order;
	\item \modexpBlakePhase{}:
		``phase'' column;
	\item \index{}:
		index column
	\item \indexMax{}:
		associated ``maximum index'' column;
\end{enumerate}
A data limb is uniquely determined by
(\emph{a}) a unique identifier,
(\emph{b}) a phase and
(\emph{c}) an index.
These are provided above.

\saNote{} The present module only stores data with size which is known ahead of time. It therefore doesn't require an \nBytes{} column.

The following column is a ``limb'' which contains limbs (i.e. $\llarge$-byte integers) representing data.
This data may be either input data
(\inst{MODEXP} base, exponent or modulus, \inst{BLAKE2f} ``\col{rounds}'' and ``\col{f}'' parameters and \col{h}, \col{m} and \col{t} parameters)
or ouput data
(\inst{MODEXP} result or \inst{BLAKE2f} result.)
\begin{enumerate}[resume]
	\item \limb{}:
		data limb column;
\end{enumerate}
The following columns are exclusive binary columns that allow the present module to identify the nature of the data it holds.
\begin{multicols}{2}
	\begin{enumerate}[resume]
		\item \isModexpBase{}
		\item \isModexpExponent{}
		\item \isModexpModulus{}
		\item \isModexpResult{}
		\item \isBlakeData{}
		\item \isBlakeParams{}
		\item \isBlakeResult{}
		\item[\vspace{\fill}]
	\end{enumerate}
\end{multicols}
