We enforce the following constraints:
\begin{enumerate}
    % \item \If $\internalChecksPassed_{i} = 0$ \Then $\trivialPairing_{i} = 0$
    % \item \If $\notOnGTwoAccMax_{i}      = 1$ \Then $\trivialPairing_{i} = 0$
    \item \trivialPairing{} is \textbf{counter-constant binary} \quad (\trash)
    \item \If $\isEcpairingData _{i} = 0$ \Then $\trivialPairing _{i} = 0$
    \item \If $\isEcpairingData_{i - 1} = 0$ \et $\isEcpairingData _{i} = 1$ \Then $\trivialPairing_{i} = 1$
    \item \If $\locTransitionFromLargeToSmall_{i} = 1$ \Then $\trivialPairing_{i+1} = \trivialPairing_{i}$
    \item  \If $\locTransitionFromSmallToLarge_{i} = 1$ \Then
          \begin{enumerate}
              \item \If $\trivialPairing _{i} = 0$ \Then $\trivialPairing _{i + 1} = 0$
              \item \If $\trivialPairing _{i} = 1$ \Then $\trivialPairing _{i + 1} = \isInfinity_{i + 1}$
          \end{enumerate}
\end{enumerate}
\saNote{} The above means that \trivialPairing{} records whether or not the pairing data contains at least one large (supposed $G_2$) point that isn't the point at infinity.

We use the \trivialPairing{} flag to set the result of unexceptional yet trivial pairings:
\begin{enumerate}[resume]
    \item we impose that
          \[
              \If
              \left\{ \begin{array}{lcl}
                  \locTransitionToResult _{i} & = & 1 \\
                  \internalChecksPassed  _{i} & = & 1 \\
                  \notOnGTwoAccMax       _{i} & = & 0 \\
                  \trivialPairing     _{i}    & = & 1 \\
              \end{array} \right.
              \Then
              \left\{ \begin{array}{lcl}
                  \locPairingsResultHi  & \define & \ecdataLimb _{i + 1} \\
                  \locPairingsResultLo  & \define & \ecdataLimb _{i + 2} \\
                  \ecdataSuccessBit_{i} & =       & 1                    \\
                  \locPairingsResultHi  & =       & 0                    \\
                  \locPairingsResultLo  & =       & 1
              \end{array} \right.
          \]
\end{enumerate}
