The following are the columns of the shifting module. The first collection of columns determines the heartbeat.
\begin{enumerate}
	\item $\shfSTAMP$: imported column containing the shifting module stamp; grows monotonically; abbreviatged to $\shfStamp$
	\item \iomf{}:
	bit column, is zero when the stamp is zero, one when the stamp is not zero; 
	\item $\CT$: column which either hovers around 0 or grows monotonically from $0$ to $\llargeMO$ and resets to $0$; abbreviated to $\ct$;
	\item $\OLI$: counter-constant binary column; determines if an instruction can be dealt with in one line or requires $\llarge$ lines to process; abbreviated to $\oli$;
\end{enumerate}
The first argument $\argOne = \col{s}$ of a shift instruction is the integer by which one shifts, the second argument $\argTwo = \col{v}$ is the value which one shifts. For instance: $\inst{SHL}$ computes $\col{v}\ll\col{s}$ and \inst{SHR} computes $\col{v}\gg\col{s}$. The result is recorded in $\res$.
\begin{enumerate}[resume]
	\item $\argOneHi$, $\argOneLo$: high and low part of the first argument of a shift instruction;
	\item $\argTwoHi$, $\argTwoLo$: high and low part of the second argument of a shift instruction;
	\item $\resHi$, $\resLo$: high and low part of the result of executing the opcode;
\end{enumerate}
Given the stack pattern used for shift-instructions (i.e. $\stdSP$)
$\argOneHi$ and $\argOneLo$ are imports of $\sValHi{1}$ and $\sValLo{1}$, 
$\argTwoHi$ and $\argTwoLo$ are imports of $\sValHi{3}$ and $\sValLo{3}$, 
$\resHi$ and $\resLo$ are imports of $\sValHi{4}$ and $\sValLo{4}$ respectively.
\begin{enumerate}[resume]
	\item $\INST$: imported instruction column;
	\item $\SHD$: binary column; abbreviated to $\shd$
	\item $\someBits$: binary column;
	\item $\NEG$: counter-constant binary column;
	\item $\known$: counter-constant binary column;
\end{enumerate}
When $\oli = 1$ the \someBits{} column is made to contain the bit decomposition of the most significant byte of $\argTwo$ and the least significant byte of $\argOne$. The most significant bit of $\argTwo$ is stored in the (counter-constant binary) column $\NEG$: when $\argTwo$ is interpreted as a signed integer, as in the case of a \inst{SAR} instruction, $\NEG$ thus contains the sign bit. If $\argOne$ differs from its least significant byte the result of applying \inst{SHR} or \inst{SHL} will be zero. The binary column $\known$ determines how $\resHi$ and $\resLo$ are computed: when $\known = 1$ the result of the shift is ``known in advance.''
\begin{enumerate}[resume]
	\item $\laHi$, $\raHi$, $\laLo$, $\raLo$: instruction decoded byte columns;
	\item $\LT$: counter-constant columns with values in $\{0,1,\dots,7\}$;
	\item $\MSHP$: counter-constant column with values in $\{0,1,\dots,7,8\}$; abbreviated to $\mshp$;
\end{enumerate}
The interpretation is as follows:
$\LT$ records the 3 least significant bits of $\argOne$;
$\mshp$ equals either $\LT$ or to $8 - \LT$ depending on whether the instruction shifts to the right or to the left;
$\laHi$, $\raHi$ are left aligned suffixes and right aligned prefixes of $\argTwoHi$ respectively;
$\laLo$, $\raLo$ are left aligned suffixes and right aligned prefixes of $\argTwoLo$ respectively.
Thus $\argTwoHi$ and $\argTwoLo$ are split in half at the bit index $\mshp$ and the resulting prefix and suffix are right-shifted and left-shifted respectively.
\begin{enumerate}[resume]
	\item $\bit{k}$ for $k \in \{1, 2, 3, 4\}$: counter-constant binary columns
	\item $\bit{b_{k}}$ for $k \in \{3, 4, 5, 6, 7\}$: counter-constant binary columns
\end{enumerate}
We introduce some byte columns:
\begin{enumerate}[resume]
	\item $\byteCol{k}$, $k=1,2,3,4,5$: byte columns for byte accumulation;
	\item $\acc{k}$, $k=1,2,3,4,5$: the corresponding accumulator columns;
	\item $\shiftedBytes{k}\high, \shiftedBytes{k}\low$ for $k=3,4,5,6,7$: byte columns that contain the result of progressive shifts;
\end{enumerate}
\noindent Note that the ``final'' shifted bytes column one might expect from the arithmetization below is $\byteCol{5}$
