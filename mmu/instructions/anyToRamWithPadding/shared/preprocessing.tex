\begin{center}
	\boxed{%
		\text{The pre-processing presented below assumes that }
		\left\{ \begin{array}{lcl}
			\isMacro                         _{i}      & = & 1 \\
			\mmuInstFlagAnyToRamWithPadding  _{i}      & = & 1 \\
		\end{array} \right.
		}
\end{center}
Recall that the \mmuMod{} module's processing of the \mmuInstAnyToRamWithPadding{} is multiplexed into two subcases, see note~(\ref{mmu: decoding: any to ram with padding multiplexing}).
We outline the common preprocessing rows
\begin{description}
	\item[Number of preprocessing rows:]
		we impose $\ppTotLZ _{i} = 0$
	\item[Some shorthands ``target range'' shorthands:]
		we define
		\[
			\left\{ \begin{array}{lcl}
				\locMinTgtOffset & \define & \macroTgtOffsetLo_{i}                        \\
				\locMaxTgtOffset & \define & \macroTgtOffsetLo_{i} + (\macroSize_{i} - 1) \\
			\end{array} \right.
		\]
	\def\rowNum{1} \item[Processing row $n^\circ(i + \rowNum)$:]
		we impose
		\[
			\left\{ \begin{array}{lcl}
				\callToLt
				{i}{\rowNum}
				{\macroSrcOffsetHi_{i}}{\macroSrcOffsetLo_{i}}
				{\macroRefSize_{i}}
				\vspace{2mm} \\
				\callToEuc
				{i}{\rowNum}
				{\locMinTgtOffset}
				{\llarge}
				\\
			\end{array} \right.
		\]
		and we define the following shorthands
		\[
			\left\{ \begin{array}{lcl}
				\locPurePadding & \define & 1 - \ppWcpRes   _{i + \rowNum} \\
				\locMinTlo      & \define & \ppEucQuot      _{i + \rowNum} \\
				\locMinTbo      & \define & \ppEucRem       _{i + \rowNum} \\
			\end{array} \right.
		\]
		\begin{enumerate}
			\item \If $\locPurePadding = 1$ \Then $\locMaxSrcOffsetOrZero \define 0$
			\item \If $\locPurePadding = 0$ \Then $\locMaxSrcOffsetOrZero \define \macroSrcOffsetLo_{i} + \big( \macroSize_{i} - 1 \big)$
		\end{enumerate}
	\def\rowNum{2} \item[Processing row $n^\circ(i + \rowNum)$:]
		we impose
		\[
			\left\{ \begin{array}{lcl}
				\callToLt
				{i}{\rowNum}
				{0}{\locMaxSrcOffsetOrZero_{i}}
				{\macroRefSize_{i}}
				\vspace{2mm} \\
				\callToEuc
				{i}{\rowNum}
				{\locMaxTgtOffset}
				{\llarge}
				\\
			\end{array} \right.
		\]
		and we \textbf{unconditionally} define the following shorthands
		\[
			\left\{ \begin{array}{lclr}
				\locMixed    & \define & (1 - \locPurePadding) \cdot (1 - \ppWcpRes_{i + \rowNum}) \\
				\locPureData & \define & (1 - \locPurePadding) \cdot \ppWcpRes_{i + \rowNum}       \\
				\locMaxTlo   & \define & \ppEucQuot      _{i + \rowNum}                            \\
				\locMaxTbo   & \define & \ppEucRem       _{i + \rowNum}                            \\
			\end{array} \right.
		\]
		At this stage it also makes sense to define the \locTransferSize{} and \locPaddingSize{} shorthands that will be used later
		\begin{enumerate}
			\item \If $\locPurePadding = 1$ \Then 
				\[ 
				\left\{ \begin{array}{lcl}
				\locTransferSize & \define & 0              \\
				\locPaddingSize  & \define & \macroSize_{i} \\
				\end{array} \right.
				\]
			\item \If $\locMixed       = 1$ \Then 
				\[ 
				\left\{ \begin{array}{lcl}
				\locTransferSize & \define & \macroRefSize_{i} - \macroSrcOffsetLo_{i}                    \\
				\locPaddingSize  & \define & \macroSize_{i} - (\macroRefSize_{i} - \macroSrcOffsetLo_{i}) \\
				\end{array} \right.
				\]
			\item \If $\locPureData = 1$    \Then 
				\[ 
				\left\{ \begin{array}{lcl}
				\locTransferSize & \define & \macroSize_{i} \\
				\locPaddingSize  & \define & 0              \\
				\end{array} \right.
				\]
		\end{enumerate}
		\saNote{} These values satisfy the following:
		\[
		\left\{ \begin{array}{lclcl}
				0 & \leq & \locTransferSize & \leq & \min\Big\{ \macroSize_{i}, \macroRefSize_{i} \Big\} \\
				0 & \leq & \locPaddingSize  & \leq & \macroSize_{i} \\
				\multicolumn{5}{l}{\locTransferSize + \locPaddingSize  =  \macroSize_{i}} \\
			\end{array} \right.
		\]
	\item[Justifying the execution path:]
		we impose the following constraints:
		\[
			\left\{ \begin{array}{lcl}
				\mmuInstFlagAnyToRamWithPaddingPurePadding  _{i} & = & \locPurePadding \\
				\mmuInstFlagAnyToRamWithPaddingSomeData     _{i} & = & \locMixed + \locPureData    \\
			\end{array} \right.
		\]
\end{description}
From this point on the analysis follows different paths according to the execution path which was justified above.
